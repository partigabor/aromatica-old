\subsection{Myrrh}

% English,myrrh
% Middle English,mirre
% Old English,myrra
% Germanic,?
% Old French,mirre (myrrhe)
% Latin,myrrha
% Ancient Greek,múrrā
% Semitic,*mrr




% EE:
% myrrh 

% gum resin from trees of genus Commiphora. OE. myrra, myrre, corr. to OS. myrra (Du. mirre), OHG. myrra (G. myrrhe), ON. mirra; Gmc. — L. myrrha — Gr. múrrā, of Sem. orig. (cf. Arab. murr, Aram. mūrā); reinforced in ME. from OF. mirre (mod. myrrhe).

% OE:
% myrrh (n.)

% "gummy, resinous exudation of certain plants of Arabia and Ethiopia," used for incense, perfumery, etc., Middle English mirre, from Old French mirre (11c.) and also from Old English myrre, both the Old English and Old French words from Latin myrrha (source also of Dutch mirre, German Myrrhe, French myrrhe, Italian, Spanish mirra), from Greek myrrha, from a Semitic source (compare Akkadian murru, Hebrew mor, Arabic murr "myrrh"), from a root meaning "was bitter." The classical spelling restoration is from 16c.

% MW:
% Middle English myrre, mirre, from Old English myrre, myrra, from Latin murra, murrha, myrrha, from Greek myrrha, of Semitic origin; akin to Hebrew mōr myrrh, mar bitter, Arabic murr myrrh, bitter

% First Known Use: before 12th century (sense 1)

% AH:
% [Latin myrrhis, an aromatic umbellifer (perhaps sweet cicely), from Greek murrhis, from murrha, myrrh (resin from trees of the genus Commiphora); see MYRRH1.]

% WK:
% From Middle English mirre, from Old English myrre, from Latin myrrha, from Ancient Greek μύρρα (múrrha), from Semitic. Compare Arabic مُرّ (murr, “myrrh”, literally “bitterness”), Hebrew מור \ מֹר (mōr, “myrrh”, literally “bitterness”). Compare מרור : maror. 

