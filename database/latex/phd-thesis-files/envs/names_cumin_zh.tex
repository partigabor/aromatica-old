\begin{table}[!ht]
\centering
\begin{tabularx}{\textwidth}{@{}l>{\itshape \small}ll>{\itshape}lL>{\small}l@{}}
\toprule
\textbf{\#} & \multicolumn{1}{l}{\textbf{Species}} & \multicolumn{1}{l}{\textbf{Name}} & \multicolumn{1}{l}{\textbf{Tr.}} & \multicolumn{1}{l}{\textbf{Gloss}} & \multicolumn{1}{l}{\textbf{Source}} \\
\midrule
1	& Cuminum cyminum	& \traditionalchinesefont{茴香籽}	& huíxiāngzǐ	& hui-spice-seed	& \textcite{mdbg} \\
2	& Cuminum cyminum	& \traditionalchinesefont{枯茗}	& kūmíng	& withered-tea	& \textcite{mdbg} \\
3	& Cuminum cyminum	& \traditionalchinesefont{羅馬葛縷子}	& luómǎ gě​lǚ​zi	& Roman-caraway	&  \\
4	& Cuminum cyminum	& \traditionalchinesefont{馬芹子}	& mǎqínzi	& horse-celery-seed	&  \\
5	& Cuminum cyminum	& \traditionalchinesefont{蒔蘿}	& shíluó	& dill-turnip	& \textcite{laufer_sino-iranica_1919} \\
6	& Cuminum cyminum	& \traditionalchinesefont{小茴香}	& xiǎohuíxiāng	& small-hui-spice-seed	& \textcite{laufer_sino-iranica_1919} \\
\textbf{7}	& \textbf{Cuminum cyminum}	& \textbf{\traditionalchinesefont{孜然}}	& \textbf{zīrán}	& \textbf{}	& \textbf{\textcite{mdbg}} \\
8	& Cuminum cyminum	& \traditionalchinesefont{孜然芹}	& zī​ránqín	& cumin-celery	& \textcite{hu_food_2005} \\
9	& Cuminum cyminum	& \traditionalchinesefont{阿拉伯茴香}	& ālābó huíxiāng	& Arabian fennel	& \textcite{mdbg} \\
10	& Cuminum cyminum	& \traditionalchinesefont{安息茴香}	& ānxī huíxiāng	& Parthian fennel	& \textcite{mdbg} \\
11	& Cuminum cyminum	& \traditionalchinesefont{歐蒔蘿}	& ōu​ shí​luó	& European dill	& \textcite{mdbg} \\
\bottomrule
\end{tabularx}
\caption{Various names for cumin in Chinese.}
\label{table:names_cumin_zh}
\end{table}

