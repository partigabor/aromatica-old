\begin{table}[!ht]
\centering
\begin{tabularx}{\textwidth}{@{}ll>{\itshape}lLl>{\small}l@{}}
\toprule
\textbf{\#} & \textbf{Language} & \multicolumn{1}{l}{\textbf{Term}} & \textbf{Gloss} & \textbf{Loan} & \multicolumn{1}{l}{\textbf{Source}} \\
\midrule
1	& English	& curcuma	& 	& yes	& \textcite{oed} \\
2	& English	& Indian saffron	& 	& no	& \textcite{oed} \\
3	& English	& turmeric	& 	& yes	& \textcite{oed} \\
\midrule
1	& Arabic	& hurd	& 	& yes	& \textcite{lane_arabic-english_1863} \\
2	& Arabic	& kurkum	& 	& yes	& \textcite{wehr_dictionary_1976} \\
3	& Arabic	& ʿuqda ṣafrā'	& yellow knob	& no	& \textcite{baalbaki_-mawrid_1995} \\
\midrule
1	& Chinese	& huángjiāng	& yellow-ginger	& no	& \textcite{defrancis_abc_2003} \\
2	& Chinese	& jiānghuáng	& ginger-yellow	& no	& \textcite{kleeman_oxford_2010} \\
\bottomrule
\end{tabularx}
\caption{Conventionalized names for turmeric in English, Arabic, and Chinese, found in dictionaries.}
\label{table:names_turmeric}
\end{table}

