\begin{table}[!ht]
\centering
\begin{tabularx}{\textwidth}{@{}ll>{\itshape}lLl>{\small}l@{}}
\toprule
\textbf{\#} & \textbf{Language} & \multicolumn{1}{l}{\textbf{Term}} & \textbf{Gloss} & \textbf{Loan} & \multicolumn{1}{l}{\textbf{Source}} \\
\midrule
1	& English	& allspice	& 	& no	& \textcite{oed} \\
2	& English	& Jamaica pepper	& 	& no	& \textcite{oed} \\
3	& English	& pimento	& 	& yes	& \textcite{oed} \\
4	& English	& pimento berry	& 	& no	& \textcite{oed} \\
5	& English	& pimiento	& 	& yes	& \textcite{oed} \\
\midrule
1	& Arabic	& fulful al-basātīn	& pepper of the gardens	& no	& \textcite{almaany} \\
2	& Arabic	& fulful ifranjī	& European pepper	& no	& \textcite{baalbaki_-mawrid_1995} \\
3	& Arabic	& fulful tābil	& spice pepper	& no	& \textcite{almaany} \\
4	& Arabic	& fulful ḥulw	& sweet pepper	& no	& \textcite{baalbaki_-mawrid_1995} \\
\midrule
1	& Chinese	& duōxiāngguǒ	& many-spice-fruit	& no	& \textcite{kleeman_oxford_2010} \\
2	& Chinese	& tiánhújiāo	& sweet-barbarian-pepper	& no	& \textcite{yellowbridge} \\
3	& Chinese	& yámǎijiā hújiāo	& Jamaica-barbarian-pepper	& yes	& \textcite{mdbg} \\
4	& Chinese	& zhòngxiāngzǐ	& many-spice-seed	& yes	& \textcite{mdbg} \\
\bottomrule
\end{tabularx}
\caption{Conventionalized names for allspice in English, Arabic, and Chinese, found in dictionaries.}
\label{table:names_allspice}
\end{table}

