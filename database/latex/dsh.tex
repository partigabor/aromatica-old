\documentclass{dsh} % In draft, initial, or final modes
% \usepackage[final]{microtype}

% "Spice Linguistics: Tracing the Linguistic Diffusion of Asian Spice Names in English"
% "A Linguistic Journey through Spice Routes: The Etymology of Asian Spice Names in English"
% "From Asia to English Tables: Unraveling the Linguistic Evolution of Spice Names"
% "The Spice of Words: A Linguistic Exploration of Asian Spice Names in English"
% "Flavorful Transitions: Investigating the Linguistic Diffusion of Asian Spices into English"
% "Savoring Language: Mapping the Linguistic Odyssey of Asian Spice Names in English"
% "Tales of Taste and Language: A Linguistic Analysis of English Spice Names' History and Diffusion"
% "Spices in Words: An Etymological Study of Asian Spice Names in the English Language"



\title{Mapping the Language of Spices: The Linguistic and Geographic Diffusion of Spices}
\author[1]{\orcid{0000-0003-2042-4655}~Gábor Parti\footnote{Hong Kong}}
\author[2]{\orcid{0000-0002-8526-5520}~Chu-Ren Huang}
\affil[1,2]{The Hong Kong Polytechnic University}
\date{\small{November 2023}}

\begin{document}

\maketitle

\begin{abstract}
\end{abstract}

\keywords{one, two, three}

\section{Introduction}

Spices in the historical sense not only amount to dried plant parts, they also constitute a set of cultural items. 






only comprise the aromatic parts of plants, but also include other plant products that were used for flavoring, preserving, and coloring food. In this paper, we will focus on the linguistic and geographic diffusion of spices, referring to how various spice names have spread around the globe - together with the materials/products. The work is philological, and it will focus on the etymologies of 24 spice words in English, e.g.: pepper, cumin, cardamom, coriander, ginger, clove, turmeric, cinnamon, etc. We will trace these words and through this, we can track the journey of the materials. We also want to refer to related historical times and civilizations that acted as donor languages in these words' history, e.g. Arabia, Southeast Asia, India, etc. Our main hypothesis is that along their diffusion in early global trade, these materials have left a continuous linguistic trail, which then serves as evidence for their journey even at time-depths where written or archeological evidence is not available. The research question would be: can looking at the words help and tell us the history and journey really? The conclusion then would be something about how our world and foods are interconnected for thousands of years.







Introduction:
Present the context of your research in Digital Humanities and the significance of studying spice word etymologies.
Introduce the research question and the main hypothesis.
Briefly outline the structure of the paper.

\dots

Certainly, the "Introduction" section of your academic paper serves as the roadmap for your readers, providing essential context and setting the stage for the rest of your research. Here's a more detailed breakdown of what to include in the introduction:

    General Introduction to Digital Humanities:
        Begin with a broad introduction to the field of Digital Humanities. Explain what Digital Humanities is and its significance in the modern academic landscape. You can define Digital Humanities as the intersection of technology and the humanities, emphasizing how it enables new and innovative approaches to traditional research questions.

    Relevance of Your Research:
        Discuss why your research is relevant within the Digital Humanities framework. Explain how studying linguistic and geographic diffusion of spices through digital methods and etymology analysis aligns with the goals and methods of Digital Humanities.

    Research Background:
        Provide a concise overview of the history of spice trade, the role of spices in global commerce, and their cultural and culinary significance. This background information helps readers understand the importance of your specific study.

    Statement of the Research Question:
        Clearly state your research question, which, in your case, is something like, "Can looking at the etymologies of spice words tell us about the history and journey of spices in early global trade?"

    Main Hypothesis:
        Present your main hypothesis, which you've mentioned: "Along their diffusion in early global trade, these materials have left a continuous linguistic trail, which then serves as evidence for their journey even at time-depths where written or archaeological evidence is not available."

    Structure of the Paper:
        Give readers a preview of how the paper is organized. Provide a brief overview of the sections that follow, explaining what each section will cover.

By following this structure for your introduction, you will provide your readers with a clear understanding of the context, importance, and objectives of your research, setting the stage for the detailed analysis and findings that follow in the subsequent sections of your paper.

***********************************************************

Question

I would like to ask for advice on how to structure an academic paper. The field is Digital Humanities, and the targeted journal is called Scholarship in the Digital Humanities. I want to write about the linguistic (and geographic) diffusion of spices, referring to how various spice names have spread around the globe - together with the materials/products. The work is philological, and it will focus on the etymologies of 24 spice words in English, e.g.: pepper, cumin, cardamom, coriander, ginger, clove, turmeric, cinnamon, etc. I will trace these words and through this, we can track the journey of the materials. I also want to refer to related historical times and civilizations that acted as donor languages in these words' history, e.g. Arabia, Southeast Asia, India, etc. Our main hypothesis is that along their diffusion in early global trade, these materials have left a continuous linguistic trail, which then serves as evidence for their journey even at time-depths where written or archeological evidence is not available. The research question would be: can looking at the words help and tell us the history and journey really? The conclusion then would be something about how our world and foods are interconnected for thousands of years. How would I structure such a paper?

Answer

Structuring an academic paper for a journal in Digital Humanities, like "Scholarship in the Digital Humanities," to investigate the linguistic and geographic diffusion of spices, with a focus on etymologies of spice words, is a complex task. Here's a suggested structure for your paper:

    Title: Start with a clear, concise, and engaging title that reflects the main topic of your research, such as "Tracing the Linguistic and Geographic Diffusion of Spices: A Philological Exploration of Spice Etymologies."

    Abstract: Provide a brief summary of your paper, including the research question, methods, key findings, and implications. This should be around 150-250 words.

    Introduction:
        Present the context of your research in Digital Humanities and the significance of studying spice word etymologies.
        Introduce the research question and the main hypothesis.
        Briefly outline the structure of the paper.

    Literature Review:
        Discuss existing literature on spice diffusion, the history of spice trade, linguistic anthropology, and etymology studies.
        Highlight gaps in the literature that your research aims to address.

    Methodology:
        Explain your research methods, focusing on how you traced the etymology of 24 spice words.
        Describe the linguistic and historical databases, tools, and resources you used for your analysis.
        Discuss how you intend to use maps and visualizations in your research.

    Data Collection:
        Present the 24 spice words you are studying and their etymological information.
        Explain your criteria for selecting these words.
        Describe how you collected linguistic data for each word and any challenges encountered during the process.

    Analysis:
        Explore the historical and linguistic journey of each spice word, tracing its origins and the languages it passed through.
        Discuss the historical contexts and civilizations that contributed to the diffusion of these words.
        Present any patterns or trends that emerge from your analysis.

    Visualizations:
        Include maps and other visual representations of the linguistic diffusion of spices. Explain the significance of these visualizations and how they support your findings.

    Discussion:
        Interpret the results of your analysis in the context of your research question and hypothesis.
        Discuss the implications of your findings for understanding the history and interconnectedness of global trade and cuisine.
        Reflect on the limitations of your study and potential areas for further research.

    Conclusion:

    Summarize the main findings and their significance.
    Reiterate how the study of spice word etymologies can serve as evidence for the historical journey of materials.
    Emphasize the broader implications of your research for understanding the interconnectivity of cultures and cuisines.

    References:

    Cite all the sources you referenced in your paper using APA style.

    Appendices (if necessary):

    Include additional details, such as extended data tables, supplementary maps, or linguistic data sources.

    Acknowledgments (optional):

    Thank individuals, institutions, or organizations that provided support for your research.

Make sure to adhere to the guidelines and formatting requirements of the "Scholarship in the Digital Humanities" journal, and consider seeking feedback from peers or mentors in your field to ensure the paper's quality and clarity.










Our main hypothesis is that along their diffusion in early global trade, these materials have left a continuous lingusitic trail, which then serves as evidence for their journey even at time-depths where written or archeological evidence is not available.



Context Setting: Introduce the significance of studying linguistic diffusion, especially in the realm of spice names.
Research Objective: Clearly state the goal of the study, which is to trace the history and etymology of various Asian spice names in English and analyze how linguistic journey informs us about physical/geographical diffusion.
Literature Review: Discuss relevant literature in digital humanities, linguistics, and culinary history to provide a background and rationale for your research.

\section{Methodology}

Linguistic Analysis: Describe the linguistic methods and tools used for analyzing the etymology and historical linguistic changes of spice names.
Data Collection: Explain the sources of data, including linguistic corpora, historical texts, dictionaries, and any specialized linguistic databases used.
Data Processing: Outline how the collected data was processed and prepared for analysis.

\section{Data}

Selection Criteria: Detail the criteria used for selecting specific Asian spices for this study.
Spice Names and Variations: Present a comprehensive list of Asian spice names in their original languages and their English variations.

\section{Results}

Etymology Analysis: Analyze the etymology of each spice name, tracing its linguistic journey and any notable changes in pronunciation and meaning.
Geographical and Historical Analysis: Discuss the geographical and historical context of how these spice names spread and were adopted into the English language.

5. Maps and Visualizations

Linguistic Diffusion Map: Create a map illustrating the diffusion of spice names across regions, showing linguistic transitions and variations.
Historical Timeline Visualization: Develop a timeline visualization highlighting the key historical events correlating with the adoption of specific spice names in English.

\section{Conclusion}

Summary of Findings: Recap the key findings regarding the linguistic journey and geographical diffusion of Asian spice names into English.
Interpretation and Implications: Discuss the implications of linguistic diffusion on our understanding of culinary history, cultural exchange, and migration patterns.
Future Research Directions: Suggest potential areas for further research, including exploring other languages' influence on English spice names and their cultural implications.

7. Acknowledgments and References

Acknowledgments: Express gratitude to individuals, organizations, or institutions that supported your research.
References: Provide a comprehensive list of all sources referenced and cited in your paper.

8. Appendices

Glossary: Include a glossary of terms and translations used in the paper for readers' convenience.

To tell a "spicy" story in your paper, consider weaving in historical anecdotes, linguistic transformations, and cultural contexts associated with each spice. Engage the reader by vividly describing the linguistic evolution and how it mirrors the spice's journey across cultures and regions. Incorporate intriguing facts and engaging language to add flavor to your narrative, making the paper not just informative but also enjoyable to read.



\printbibliography

\theendnotes

\end{document}
