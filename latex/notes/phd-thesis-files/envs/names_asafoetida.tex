\begin{table}[!ht]
\centering
\begin{tabularx}{\textwidth}{@{}ll>{\itshape}lLl>{\small}l@{}}
\toprule
\textbf{\#} & \textbf{Language} & \multicolumn{1}{l}{\textbf{Term}} & \textbf{Gloss} & \textbf{Loan} & \multicolumn{1}{l}{\textbf{Source}} \\
\midrule
1	& English	& devil's dung	& 	& no	& \textcite{oed} \\
2	& English	& hing	& 	& yes	& \textcite{oed} \\
3	& English	& asafoetida	& 	& yes	& \textcite{oed} \\
\midrule
1	& Arabic	& abū kabīr	& big father	& no	& \textcite{wehr_dictionary_1976} \\
2	& Arabic	& anjudān	& 	& yes	& \textcite{baalbaki_-mawrid_1995} \\
3	& Arabic	& samgh al-anjudān	& gum of anjudan	& no	& \textcite{baalbaki_-mawrid_1995} \\
4	& Arabic	& samgh rātīnājī	& rātīnājī gum	& no	& \textcite{baalbaki_-mawrid_1995} \\
5	& Arabic	& ḥiltīt	& 	& yes	& \textcite{wehr_dictionary_1976} \\
\midrule
1	& Chinese	& āwèi	& 	& yes	& \textcite{mdbg} \\
\bottomrule
\end{tabularx}
\caption{Conventionalized names for asafoetida in English, Arabic, and Chinese, found in dictionaries.}
\label{table:names_asafoetida}
\end{table}

