\begin{spice}\label{spice:cinnamon}
\textsc{Cinnamon} \hfill \href{https://powo.science.kew.org/taxon/463752-1}{POWO} \\
\textbf{English:} \textit{cinnamon}. 
\textbf{Arabic:} {\arabicfont{قرفة}} \textit{qirfa} [rind; bark]; {دارصيني} \textit{dārsīnī}. 
\textbf{Chinese:} {\traditionalchinesefont{錫蘭肉桂}} \textit{xīlánròuguì} [Ceylon-flesh-cinnamon]. 
\textbf{Hungarian:} \textit{fahéj} [tree-bark].  \\
\noindent{\color{black}\rule[0.5ex]{\linewidth}{.5pt}}
\begin{tabular}{@{}p{0.25\linewidth}@{}p{0.75\linewidth}@{}}
Plant species: & \taxonn{Cinnamomum verum}{J.Presl.} (syn. \taxonn{Cinnamomum zeylanicum}{Blume}) \\
Family: & \textit{Lauraceae} \\
part used: & bark; leaf \\
Region of origin: & Sri Lanka; SW. India \\
Cultivated in: & Sri Lanka; Seychelles; Madagascar; India \\
Color: & warm yellowish-brown, cinnamon \sample{cinnamon} \\
\end{tabular}
\end{spice}