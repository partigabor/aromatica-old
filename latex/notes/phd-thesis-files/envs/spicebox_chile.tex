\begin{spice}\label{spice:chile}
\textsc{Chile} \hfill \href{https://powo.science.kew.org/taxon/316944-2}{POWO} \\
\textbf{English:} \textit{chile}; \textit{chilli; chili pepper}. 
\textbf{Arabic:} {\arabicfont{فلفل حار}} \textit{fulful hārr} [hot pepper]. 
\textbf{Chinese:} {\traditionalchinesefont{辣椒}} \textit{làjiāo} [pungent-pepper]. 
\textbf{Hungarian:} \textit{paprika}; \textit{pirospaprika} [red-pepper]; \textit{fűszerpaprika} [spice-pepper]; \textit{erős-paprika} [strong-pepper]; \textit{csilipaprika} [chili-pepper]; \textit{Cayenne bors} [Cayenne pepper]; \textit{törökbors} [Turkish-pepper] (historic).  \\
\noindent{\color{black}\rule[0.5ex]{\linewidth}{.5pt}}
\begin{tabular}{@{}p{0.25\linewidth}@{}p{0.75\linewidth}@{}}
Plant species: & \taxonn{Capsicum annuum}{L.}; \textit{\taxonn{Capsicum frutescens}{L.}; \taxonn{Capsicum chinense}{Jacq.}; et al.} \\
Family: & \textit{Solanaceae} \\
part used: & fruit \\
Region of origin: & Central America \\
Cultivated in: & Ethiopia; India; Kenya; Mexico; Nigeria; Pakistan; Tanzania; etc. \\
Color: & red and green in many shades \\
\end{tabular}
\end{spice}