\begin{etymology}\label{ety:saffron}
\textbf{English} \textit{saffron}, ca. 1200; cf. Middle English \textit{saf(f)rǒun}
< \textbf{French} \textit{safran} `id.', c. 1150; cf. Middle Low German \textit{safferân}, Middle Dutch \textit{saffraen} (Dutch \textit{saffraan}), Middle High German \textit{saffrân} (modern German \textit{safran})
< \textbf{Medieval Latin} \textit{saffrānum} `id.'
< \textbf{Arabic} {زعفران} \textit{zaʿfarān} `id.', (not connected with \textit{ṣafrā'} feminine of \textit{aṣfar} `yellow'); cf. Turkish, Persian, and Hindi; Jewish Aramaic \textit{zaʿperānā}; Spanish \textit{azafran}, Portuguese \textit{açafrão}; the word without this prefix gives rise to Italian \textit{zafferano, zaffrone}, Provençal \textit{safran, safrá}, Catalan \textit{safrá}, French \textit{safran}, medieval Latin \textit{safranum}, medieval Greek ζαϕρᾶς \textit{zaforás}, modern Greek σαϕράνι \textit{safráni}, Russian \textit{šafran}. \footnote{\textcite[s.v. saffron]{oed}; \textcite[saf(f)rǒun]{med}; \textcite[s.v. safran]{tlfi}; \textcite{wehr_dictionary_1976}}
\end{etymology}