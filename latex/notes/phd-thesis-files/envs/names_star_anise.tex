\begin{table}[!ht]
\centering
\begin{tabularx}{\textwidth}{@{}ll>{\itshape}lLl>{\small}l@{}}
\toprule
\textbf{\#} & \textbf{Language} & \multicolumn{1}{l}{\textbf{Term}} & \textbf{Gloss} & \textbf{Loan} & \multicolumn{1}{l}{\textbf{Source}} \\
\midrule
1	& English	& badian	& 	& yes	& \textcite{oed} \\
2	& English	& Chinese anise	& 	& no	& \textcite{oed} \\
3	& English	& star anise	& 	& no	& \textcite{oed} \\
\midrule
\midrule
1	& Chinese	& bājiǎo	& eight-horns/octagon	& no	& \textcite{defrancis_abc_2003} \\
2	& Chinese	& bājiǎohuíxiāng	& eight-horned-hui-spice	& no	& \textcite{kleeman_oxford_2010} \\
3	& Chinese	& dàliào	& big-ingredient	& no	& \textcite{defrancis_abc_2003} \\
4	& Chinese	& dà​huíxiāng	& big-hui-spice	& no	& \textcite{mdbg} \\
\bottomrule
\end{tabularx}
\caption{Conventionalized names for star anise in English, Arabic, and Chinese, found in dictionaries.}
\label{table:names_star_anise}
\end{table}

