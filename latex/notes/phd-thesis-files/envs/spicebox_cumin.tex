\begin{spice}\label{spice:cumin}
\textsc{Cumin} \hfill \href{https://powo.science.kew.org/taxon/840882-1}{POWO} \\
\textbf{English:} \textit{cumin}. 
\textbf{Arabic:} {\arabicfont{كمون }} \textit{kammūn}. 
\textbf{Chinese:} {\traditionalchinesefont{孜然}} \textit{zī​rán}. 
\textbf{Hungarian:} \textit{római kömény} [Roman cumin].  \\
\noindent{\color{black}\rule[0.5ex]{\linewidth}{.5pt}}
\begin{tabular}{@{}p{0.25\linewidth}@{}p{0.75\linewidth}@{}}
Plant species: & \taxonn{Cuminum cyminum}{L.} \\
Family: & \textit{Apiaceae} \\
part used: & fruit \\
Region of origin: & W. \& C. Asia; India  \\
Cultivated in: & India; Iran; Lebanon \\
Color: & light brown \\
\end{tabular}
\end{spice}