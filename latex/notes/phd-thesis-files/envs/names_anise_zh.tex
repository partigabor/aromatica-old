\begin{table}[!ht]
\centering
\begin{tabularx}{\textwidth}{@{}l>{\itshape \small}ll>{\itshape}lL>{\small}l@{}}
\toprule
\textbf{\#} & \multicolumn{1}{l}{\textbf{Species}} & \multicolumn{1}{l}{\textbf{Name}} & \multicolumn{1}{l}{\textbf{Tr.}} & \multicolumn{1}{l}{\textbf{Gloss}} & \multicolumn{1}{l}{\textbf{Source}} \\
\midrule
\textbf{1}	& \textbf{Pimpinella anisum}	& \textbf{\traditionalchinesefont{茴芹}}	& \textbf{huíqín}	& \textbf{hui-celery}	& \textbf{\textcite{kleeman_oxford_2010}} \\
2	& Pimpinella anisum	& \traditionalchinesefont{茴香}	& huíxiāng	& hui-spice	& \textcite{kleeman_oxford_2010} \\
3	& Pimpinella anisum	& \traditionalchinesefont{西洋茴香}	& xīyáng huíxiāng	& western-ocean-hui-spice	& \textcite{wikipedia} \\
4	& Pimpinella anisum	& \traditionalchinesefont{洋茴香}	& yáng huíxiāng	& ocean-hui-spice	& \textcite{cec} \\
5	& Pimpinella anisum	& \traditionalchinesefont{歐洲大茴香}	& ōuzhōu dàhuíxiāng	& European-big-hui-spice	& \textcite{wikipedia} \\
\bottomrule
\end{tabularx}
\caption{Various names for anise in Chinese.}
\label{table:names_anise_zh}
\end{table}

