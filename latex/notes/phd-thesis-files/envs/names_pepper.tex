\begin{table}[!ht]
\centering
\begin{tabularx}{\textwidth}{@{}ll>{\itshape}lLl>{\small}l@{}}
\toprule
\textbf{\#} & \textbf{Language} & \multicolumn{1}{l}{\textbf{Term}} & \textbf{Gloss} & \textbf{Loan} & \multicolumn{1}{l}{\textbf{Source}} \\
\midrule
1	& English	& black pepper	& 	& maybe	& \textcite{oed} \\
2	& English	& pepper	& 	& yes	& \textcite{oed} \\
3	& English	& white pepper	& 	& maybe	& \textcite{oed} \\
\midrule
1	& Arabic	& fulful	& 	& yes	& \textcite{wehr_dictionary_1976} \\
2	& Arabic	& fulful abyaḍ	& white pepper	& no	& \textcite{baalbaki_-mawrid_1995} \\
3	& Arabic	& fulful aswad	& black pepper	& no	& \textcite{baalbaki_-mawrid_1995} \\
4	& Arabic	& fulfula	& 	& no	& \textcite{wehr_dictionary_1976} \\
\midrule
1	& Chinese	& báihújiāo	& white-barbarian-pepper	& no	& \textcite{mdbg} \\
2	& Chinese	& hújiāo	& barbarian-pepper	& no	& \textcite{defrancis_abc_2003} \\
3	& Chinese	& hēihújiāo	& black-barbarian-pepper	& no	& \textcite{mdbg} \\
\bottomrule
\end{tabularx}
\caption{Conventionalized names for pepper in English, Arabic, and Chinese, found in dictionaries.}
\label{table:names_pepper}
\end{table}

