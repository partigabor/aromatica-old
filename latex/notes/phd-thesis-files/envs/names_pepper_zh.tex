\begin{table}[!ht]
\centering
\begin{tabularx}{\textwidth}{@{}l>{\itshape \small}ll>{\itshape}lL>{\small}l@{}}
\toprule
\textbf{\#} & \multicolumn{1}{l}{\textbf{Species}} & \multicolumn{1}{l}{\textbf{Name}} & \multicolumn{1}{l}{\textbf{Tr.}} & \multicolumn{1}{l}{\textbf{Gloss}} & \multicolumn{1}{l}{\textbf{Source}} \\
\midrule
1	& Piper nigrum	& \traditionalchinesefont{白胡椒}	& báihújiāo	& white-barbarian-pepper	& \textcite{mdbg} \\
\textbf{2}	& \textbf{Piper nigrum}	& \textbf{\traditionalchinesefont{胡椒}}	& \textbf{hújiāo}	& \textbf{barbarian-pepper}	& \textbf{\textcite{hu_food_2005}} \\
3	& Piper nigrum	& \traditionalchinesefont{黑胡椒}	& hēihújiāo	& black-barbarian-pepper	& \textcite{mdbg} \\
4	& Piper nigrum	& \traditionalchinesefont{綠胡椒}	& lǜhújiāo	& green-barbarian-pepper	& \textcite{regency_spices_regency_2022} \\
5	& Piper nigrum	& \traditionalchinesefont{青胡椒}	& qīnghújiāo	& green-barbarian-pepper	& \textcite{regency_spices_regency_2022} \\
\bottomrule
\end{tabularx}
\caption{Various names for pepper in Chinese.}
\label{table:names_pepper_zh}
\end{table}

