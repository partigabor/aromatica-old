\subsection{Musk}

% English,musk
% Old French,musc
% Late Latin,muscus
% Ancient Greek,mόschos
% Persian,mušk
% Sanskrit,muṣka

% EE:
% musk 

% odoriferous substance secreted by the musk-deer. XIV. — late L. muscus — Pers. mušk, perh. — Skr. mu⋅ká — scrotum (the shape of the musk-deer's musk-bag being similar).

% OE:
% musk (n.)

% odoriferous reddish-brown substance secreted by the male musk deer (dried and used in medicinal preparations and as a perfume), late 14c., from Old French musc (13c.) and directly from Medieval Latin muscus, from Late Greek moskhos, from Persian mushk, from Sanskrit muska-s "testicle," from mus "mouse" (so called, presumably, for resemblance; see muscle). The deer gland was thought to resemble a scrotum. German has Moschus, from a Medieval Latin form of the Late Greek word. Spanish has almizcle, from Arabic al misk "the musk," from Persian.

% The musk-deer, the small ruminant of central Asia that produces the substance, is so called from 1680s. The name musk was applied to various plants and animals of similar smell, such as the Arctic musk-ox (1744). Musk-melon "the common melon" (1570s) probably originally was an oriental melon with a musky smell, the name transferred by error [OED]. Also compare Muscovy.
% Entries linking to musk
% muscle (n.)

% "contractible animal tissue consisting of bundles of fibers," late 14c., "a muscle of the body," from Latin musculus "a muscle," literally "a little mouse," diminutive of mus "mouse" (see mouse (n.)).

% So called because the shape and movement of some muscles (notably biceps) were thought to resemble mice. The analogy was made in Greek, too, where mys is both "mouse" and "muscle," and its combining form gives the medical prefix myo-. Compare also Old Church Slavonic mysi "mouse," mysica "arm;" German Maus "mouse; muscle," Arabic 'adalah "muscle," 'adal "field mouse;" Cornish logodenfer "calf of the leg," literally "mouse of the leg." In Middle English, lacerte, from the Latin word for "lizard," also was used as a word for a muscle.

%     Musclez \& lacertez bene one selfe þing, Bot þe muscle is said to þe fourme of mouse \& lacert to þe fourme of a lizard. [Guy de Chauliac, "Grande Chirurgie," c. 1425]

% Hence muscular and mousy are relatives, and a Middle English word for "muscular" was lacertous, "lizardy." Figurative sense of "muscle, strength, brawn" is by 1850; that of "force, violence, threat of violence" is 1930, American English. Muscle car "hot rod" is from 1969.
% Muscovy 

% former principality in central Russia that formed the nucleus of the modern Russian nation; from French Moscovie, from Modern Latin Moscovia, old name of Russia, from Russian Moskova "(Principality of) Moscow." In Muscovy duck (1650s), the tropical American bird, and certain other uses it is a corruption of musk. Related: Muscovite; Muscovian.

% MW:
% Middle English muske, from Middle French musc, from Late Latin muscus, from Greek moschos, from Persian mushk castoreum, musk, from Sanskrit muṣka testicle, scrotum, vulva, diminutive of mūṣ, mouse — more at mouse

% First Known Use: 14th century (sense 1a)


% AH:
% [Middle English, from Old French musc, from Late Latin muscus, from Greek moskhos, from Persian mušk, probably from Sanskrit muṣkaḥ, testicle; see mūs- in the Appendix of Indo-European roots.]

% WK:
% From Middle English muske, borrowed from Old French musc, from Late Latin muscus, from Ancient Greek μόσχος (móskhos), from Middle Persian [script needed] (mwšk' /mušk/) whence Persian مشک (mošk). Ultimately from Sanskrit मुष्क (muṣka, “testicle”), the shape of the gland of animals secreting the substance being compared to human testicles, a diminutive of मूष् (mūṣ, “mouse”), the shape of human testicles being compared to mice, from Proto-Indo-European *muh₂s (“mouse”). Cognate with mouse. 


%======================================================




