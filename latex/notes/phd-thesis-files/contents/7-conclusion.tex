\chapter{Conclusion}
\label{ch:conclusion}
\addcontentsline{toc}{chapter}{\nameref*{ch:conclusion}}
\markboth{\nameref*{ch:conclusion}}{\nameref*{ch:conclusion}}

% 9 Conclusion
% Summarize what you did, what you found and what the implications are.
% What are your answers to the research questions?
% How have you addressed your aims and objectives?

\lettrine[lines=\iniciale]{\textcolor{\accentcolor}{E}}{nding} my dissertation, I will now summarize the work that was done, reflect on the findings and their implications. Then, I will try to answer the research questions, revisit the aims stated in the beginning, and mention some of the limitations and future plans. To summarize what I did, I should start with mentioning the datasets. I have collected data in three stages, (1) on a set of spices, (2) on their names, and (3) on their related etymologies. I have done so, because from the literature it was obvious that there is a gap regarding research of spice terminology and nomenclature, which leads to chaos and confusion in the secondary sources. I hope that this dissertation is a step forward towards a future database of spice names that can be useful for both academics and the public. The datasets are unique in the sense that they incorporate botanical information, rich in philological considerations regarding sources, attestation dates, and etymological stages, and accommodate three major world languages: English, Arabic, and Chinese. I have introduced six of the 24 spices in greater detail, discussing their significance, uses, botany, history, and names (\cref{ch:data}), with the aim of us having an deeper understanding of these specific items from a comparative perspective. 

Then, the collected terms allowed for an analysis on the linguistic diffusion of spices, which I discussed in parallel with the physical diffusion of the plants and materials. I have illustrated the spatial spread of English, Arabic, and Chinese loanwords and \glspl{wanderwort} of the spice domain using interactive geospatial visualizations (\cref{ch:diffusion}). For the temporal dimension, I have illustrated the attestation dates on timelines, comparing clusters of attestation dates and borrowings across the three languages. 

My findings show that besides the interest and demand, or the volume of trade at certain times, one of the most important factors in a spice's---and its name's---successful diffusion is in their ability to \textit{survive} in new environments, alive or dead. By successful I mean widespread and long-lasting. Whether it is the enduring rhizomes of ginger and turmeric waiting to sprout on the decks of outrigger boats on the Pacific, the dried fruits of pepper capable of withstanding spoilage for years in Roman warehouses, or the easy-to-grow, ever-thriving American chile that Asians everywhere so quickly learned to cultivate, it is the ``ecological willingness'' to adapt and last which was crucial in the widespread diffusion of these plants and materials. I tried to this with a rudimentary measure I called a spreadability index. The 

% Donor and source languages

Lastly, I have looked into the various spice names, and underlying mechanisms of spice name propagation to explain how novel substances are named in this domain. I have compared the three languages by the terms' analyzability, borrowed status, and word formation strategies, and I conducted analysis on the headwords and modifiers of compound spice terms. This yielded some insights into the motivations and strategies behind the invention and derivation of spice names, and also let me see the influence of these substances highly sensory nature, marked by words of color, taste, smell, shape and form (\cref{ch:language}). 

The findings of this chapter was that on one hand the headword is usually a prototype spice, an item already known to the speakers of the language and similar in some way to the novel item to be named after it. A prototype spice word can have two roles, first on account of its similarity denoting a mismatching spice (i.e., Chinese parsley is not parsley), or with a specifying modifier a matching spice (i.e., true cardamom is cardamom). Prototype similarity can be based on any salient qualities of the substance: whether it is the chemical constituents resulting in a similar flavor (anise \rightarrow star anise), the function/use as a pungent culinary spice (pepper \rightarrow chili pepper; long pepper), or physical similarly (pepper \rightarrow Jamaica pepper). Of course, these qualities are never exclusive. On the other hand, we learned that the modifiers are most likely distinguishing words that point to a source or geographic origin (e.g., Chinese, Indian, foreign), and sensory words of color, smell, or taste that disambiguate, specify, and identify different products (e.g., black, red, green, sweet, fragrant, hot, etc.). We learned that statistically, the most typical spice name is one with a blueprint of \textit{prototype similarity headword} + \textit{geographic origin modifier}, e.g., \textit{Jamaica pepper, Indian saffron, Ethiopian cardamom}. We might call these regular spice terms, and it is interesting to notice that at the same time, these are aliases, hiding spices with different identities behind better known prototype items. Is this the reason many of us are confused by spice names? 

\subsubsection{Answers}

In \cref{sec:research_questions}, I have asked some initial research question that expected to find answers to. Q1 was ``Does the propagation of \glspl{wanderwort} within the domain of the spice trade follow the diffusion of the materials?'' The answer to this question is overwhelmingly yes, however, we found some exceptions. Notably, a few terms actually have outpaced their respective referent materials, and reached a language before the speakers were familiar with the material itself. According to my dataset of the 24 spices, this always happened via the dispersion of religious texts. In the West, the best examples are \textit{cassia} and \textit{cinnamon}, word that appear in the Bible, and with the wind of Christianity these words reached places far beyond the Eastern Mediterranean before the actual materials. In the East, our only certain example for this phenomenon is Chinese \textit{xingyu/xingqu} `hing' (asafoetida), which spread with Buddhist texts from the Silk Roads of Central Asia before Chinese monks have ever seen the gum-resin. When the product arrived a century later, it came with a new name \textit{awei} which was loaned from a Kuchean word. This is the name by which it is known today.

Q2 was ``Is there any underlying pattern behind the mechanisms of spice diffusion, considering both the materials and the nomenclature?'', and the answer was explained just above, with the findings about the diffusion of spices discussed in \cref{ch:diffusion}: The diffusion of the words strongly correlates with the diffusion of the materials---which is not surprising, but expected---and the patterns we see the etymological stages reflect the powers of trade that were situated on the crossroads between the source of the materials, and major communities with an urge for them. The findings also showed that besides the human factor, the ecology of the plants, and the materials' resistance to long-haul trade have also played a great role.

Regarding Q3: ``Is there any influence on the naming spices, in terms of sensory words and synesthetic properties?'', I have found that the sensory nature of spices plays a role in their names in two ways. One, compound spice names usually contain a headword that is a previously known prototype spice, on account of their physical (appearance, color, shape, form), and chemical (taste, aroma, heat, pungency) similarity; and two, distinguishing words are also frequently sensory words of vision, gustation, or olfaction as we saw in \cref{ch:language}. However, despite the rich sensory nature of spices, the names are most typically modified with a geographic place name. This suggests, that even if spices are colorful and aromatic, often the most salient feature is their exoticism.
    
    % \textbf{Q4} Do the presence or absence of various spice related lexical categories in a language show their level of embeddedness in a culture?
    
    % \textbf{Q5} Would the different patterns of spice name propagation and linguistic-cognitive characteristics correlate or show differences in any way?

\subsubsection{Aims Revisited}

Overall, the aims stated in the \nameref{ch:introduction} were achieved. First of all, there is a framework and a database that---although far from perfect---can be a basis for future endeavors in the research of the terminology of aromatic materials that can be interesting for philologists, linguists, historians, or culinary professionals. I would like to highlight again that this dataset is machine-readable, and after further expansion can accommodate various statistical analytics regarding the spices, and various aspects of their names. The dataset also facilitates the detailed treatise of other spices, beyond the six that were introduced in the data chapter (\nameref{ch:data}), many of which have yet to be published on from a humanities' perspective.

Our second aim was ``to map the diffusion of the terms of the spice domain'', and I hope the interactive visualizations under \cref{fig:diffusion_en,fig:diffusion_ar,fig:diffusion_zh}\footnote{Hosted on \url{https://github.com/partigabor/phd-thesis-viz}} are adequate to illustrate the significance of these wandering words. The visualizations are based on the etymological data collected and explained in the thesis, and openly available on the same repository.\footnote{Hosted on \url{https://github.com/partigabor/phd-thesis-viz/data}.}

The third goal was to shed light on how spice names are ``born'', and what are the motivations and mechanisms for spice name propagation. I feel that the findings of \cref{ch:language} achieves this goal, as we now know how the three languages borrow words, and how they devise new names based on existing words, and what are the typical components in this process.

% What is the significance and implications of your findings?
% What contribution has the study made?


\subsubsection{Significance and Contribution}

In the very beginning of this thesis I talked about a tree and its spice which has different names in various languages and turned out to be allspice. I hope that the motivation and reasons behind its many names---\textit{allspice, newspice, clove pepper, Jamaica pepper}---is not a mystery anymore. Furthermore, I hope that the next time the reader is faced with the name of a novel spice term, they can analyze it, ponder on its origins, and look up its history. I think that if we go about with open eyes and noses, we will sense the surrounding variety every day, embodied in the aromas. And the next time my grandmother is making sausage in Hungary, I can tell her that the ``clove pepper'' she is using is from the Caribbean, and even if Columbus never found it, it eventually arrived to this side of the planet; only to be confused with every other spice under the sun by the many confusing ways we named it.

The significance of this study lies in the topic; spices are ubiquitous part of our cuisine and culture. Everyone who ever had a dinner party at home---including the cooking---in a multicultural, multilingual setting knows that the nomenclature of spices and seasonings are confusing, interesting, and carry the history of the spice. My contribution is an attempt to collect, introduce, and analyze them in a way that was not done before. 

% What are the limitations of the study and questions for further research?

\section{Limitations}
\label{sec:limitations}

% \subsection{The Scope}

This project has a number of limitations. First, it was not possible to include more spices in this short time. I firmly believe that the more aromatic items we add into the fold, the more firm the results would be in terms of claiming something universal about the terms of the spice domain. The present study encompasses 24 spices and 360 names, and it feels it is just enough to start suggesting generalization about the diffusion and especially the naming of various aromatic substances.

I also maintain that there are some general truths that I can now claim, especially regarding the globally important and well-known items that were popular from Asia to Europe, and between the two endpoints of the Maritime Silk Road. Among these are the overwhelming East-to-West directionality, which is supported by the botanical reality (most of these spices are from tropical Asia), the historical evidence (the spice trade since antiquity, the Age of Discovery), and the linguistic trail (cf. the source and donor languages). My opinion, based on the data here, is that the more aromatics we add into the investigation, the proportion of the major donor languages will be similar. Words from Latin, Greek, Arabic, Persian, Akkadian, Sanskrit, Chinese, etc. and other influential languages of high culture since premodern times will ``overpower'' smaller regional languages from the source or path of the materials. There are always exceptions, of course, but the influence of languages of empires and civilizations with long-lasting influence is almost unquestionable.

Further limitations of my study arise from the fact that I tried to delimit my scope by focusing on words that arrived to English, Arabic, and Chinese, and disregarding other interesting and related cases. Although in the etymologies I ventured beyond the trilingual setting, but I now I believe that this kind of study should be aware of any language of contact, and especially influential languages of source words, e.g., Sanskrit. In any case, I believe that the trio of English, Arabic, and Chinese made for an interesting comparison for glimpse into the spice domain.

% \subsection{The Gap}

I introduced the research gap in the beginning of the thesis (\cref{sec:research_gap}), and now I believe it is impossible to fill it with one dissertation. On one hand, it is not possible to be competent in every related language, and second, it is not feasible to introduce every known spice accurately here. Maybe later, in the form of a book? I am convinced, that this project could be valuable and interesting with a team of experts that can help with the curation of relevant information.

\section{Future Studies}
\label{sec:future_studies}

In terms of future work, I am full of ideas. For future studies, I think one valuable way to dive into the diffusion of spices, incense, perfume, and other aromatics is to focus on a specific cultural area, for example the Mediterranean, or Southeast Asia. I have plans for a paper that tracks wandering food items---and their names---of East and Southeast Asia. 

% I started to prepare a study on spice names and their connection to Classical Arabic, as it was not yet collected. 

On a slightly different take diving into lexical semantics and cognitive linguistics relying on corpora and corpus linguistic tools. I have extracted words from dictionaries that derive from spice terms, and analyzed their dates of coinage and grouped them by grammatical category. Doing so, I wish to prove that the more familiar is a culture with an item, the more productive the item's name is in the languages. This study will focus on the spices' role in daily language, how spice words entered the lexicon and what is their role in metaphors and idiomatic expressions. And most importantly, how do we conceptualize spicy words, a topic that was recently touched by \textcite{bagli_tastes_2021} in English, and \textcite{dong_corpus-based_2018} and \textcite{zhong_bodily_2021} in Chinese. The goal of this is to  look at to what degree spice terminology is used in a language, which is proposed to be a gauge of a spice's embeddedness in a culture, and see how significant they are in the everyday human experience.

I also hope to extend on the attempt to collect the names for a specific spice in every significant language and plot their distributions grouped by etymon as it can be seen on \cref{sec:case_of_cinnamon}. In a more typological take, hopefully I can compare more spices similarly to find patterns and correlations. The problem with this is that while some products represent common and easy-to-collect word lists from existing databases (i.e., \textit{pepper}), more rare items would be much more difficult to gather. 

However, the most crucial task is to make this project alive and breathing. During the revision phase, I have built a website to host a spice database currently named \textit{Spice and Spice Terminology Database 1.0}, accessible at \url{https://partigabor.github.io/spice/}. I hope in the future this can be a useful portal for people to access information about spices and spice names. I will host the datasets and visualizations, which people can query, and learn about the nomenclature, etymologies, and diffusion of spice terms on the web. I plan to make this website as a sort of philologically focused spice directory, that is free and open-access. Hopefully this platform makes way to gradually update, correct, and expand on the existing data, and I also expect valuable input from other researchers in the same area, not mention the tips and guidance of people who are expert in a certain language or substance that I am not. I definitely plan to work on this topic in the long term, and I hope I can eventually collaborate with other scholars including linguists, botanists, historians, and even culinary experts to build a rich and useful database that supplies precise information.