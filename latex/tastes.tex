\documentclass[12pt]{article}
\usepackage[a4paper, margin=3cm]{geometry}
\usepackage{graphicx}
\usepackage{wrapfig}
\usepackage{subfig}
\usepackage{csquotes}
\usepackage{booktabs}
\usepackage{tabularx}
\usepackage{url}
\usepackage{soul} % strikethrough \st{}; highlight \hl{}
% \usepackage[final]{microtype}

% Referencing
\usepackage{hyperref}
\usepackage[dvipsnames]{xcolor} % Color names
\definecolor{accent}{HTML}{008080}
\hypersetup{
    colorlinks=true,
    linkcolor=accent,
    anchorcolor=black,
    citecolor=accent,
    filecolor=black,
    menucolor=black,
    runcolor=black,
    urlcolor=accent}
\hypersetup{breaklinks=true}

% Bibliography
\usepackage[
    backend=biber,
    style=apa,
    autocite=inline,
    sorting=nyt,
    sortcites=true,
    backref=false,
    backrefstyle=three,
    abbreviate=true,
    block=space,
]{biblatex}
\DefineBibliographyStrings{american}{
    backrefpage={Cited on p.},
    backrefpages={Cited on pp.}}
\apptocmd{\sloppy}{\hbadness 10000\relax}{}{}
\appto{\bibsetup}{\sloppy}
\DeclareFieldFormat{apacase}{#1} % Title casing in APA
% \AtBeginBibliography{\small} % Can make the font in bibliography smaller
\addbibresource{../static/bibliography/parti.bib} %



% Languages, scripts, and fonts
% \usepackage{libertine} % Font package: Latin, Greek, Cyrillic, Hebrew

\usepackage[main=english, bidi=basic]{babel}
\usepackage{fontspec}
\defaultfontfeatures{Ligatures={TeX}}
% \setmainfont{Brill}

% Automatic typesetting of writing systems
% \babelprovide[import=ar, onchar=ids fonts]{arabic}
% \babelprovide[import=sa-Deva, onchar=ids fonts]{sanskrit-devanagari}
% \babelprovide[import=he, onchar=ids fonts]{hebrew}

% Define fonts (https://tug.org/FontCatalogue/)
% \babelfont{rm}{Linux Libertine}
% \babelfont{sf}{Linux Biolinum}
% \babelfont{tt}{Inconsolata}

% \babelfont[*arabic]{rm}{Noto Naskh Arabic}
% \babelfont[*hebrew]{rm}{Noto Serif Hebrew}
% \babelfont[*devanagari]{rm}[Renderer=Harfbuzz]{Noto Serif Devanagari} 

% Settigs: Scale=MatchUppercase; Scale=MatchLowercase; Scale=1.0; Language=Default

% % East Asian scripts
% \usepackage{kotex} 
% \setmainhangulfont{Noto Serif KR} % Support for KR, JP, limited TC, no SC 
% \setmainhangulfont[
    % Path=./fonts/,
    % Extension = .otf,
    % UprightFont=*-Medium,
    % BoldFont=*-Bold
    % ]{NotoSerifKR}

% \newfontfamily{\traditionalchinesefont}{Noto Serif TC}

% % Other scripts
% \newfontfamily{\javanesefont}[
%     Scale=1.0,
%     Path=./fonts/,
%     Extension = .ttf,
%     ]{Javatext}

% \newfontfamily{\javanesefont}[
%     Path=./fonts/,
%     Extension = .otf,
%     UprightFont=*-Regular,
%     ]{NotoSansJavanese}

% \newfontfamily{\javanesefont}{Noto Sans Javanese}
% \newfontfamily{\cuneiformfont}{Noto Sans Cuneiform}
% \newfontfamily{\linearbfont}{Noto Sans Linear B}

% \newfontfamily{\tibetanfont}[Script=Tibetan]{Noto Serif Tibetan}
% \newfontfamily{\khmerfont}[Script=Khmer]{Noto Serif Khmer}



% % Commands to use the scripts
% \newcommand{\tc}[1]{\traditionalchinesefont{#1}\rmfamily}
% \newcommand{\bo}[1]{\tibetanfont{#1}\rmfamily}
% \newcommand{\cu}[1]{\cuneiformfont{#1}\rmfamily}
% \newcommand{\jv}[1]{\javanesefont{#1}\rmfamily}
% \newcommand{\km}[1]{\khmerfont{#1}\rmfamily}
% \newcommand{\lb}[1]{\linearbfont{#1}\rmfamily}



% Keywords command
\providecommand{\keywords}[1]{\small\noindent\textbf{\textit{Keywords ---}}#1}

% ORCiD
\definecolor{orcid}{HTML}{A6CE39}
\usepackage{academicons}
\newcommand{\orcid}[1]{\href{https://orcid.org/#1}{\textcolor{orcid}{\aiOrcid}}}

% Title, author, date
\title{The Tastes of Words: Contextualizing Cultural Contacts with Cuisine-Related Vocabulary in East Asia}
\author{Gábor Parti {\small\orcid{0000-0003-2042-4655}}}
% \author{Gábor Parti\footremember{aff}{The Hong Kong Polytechnic University}~{\small\orcid{0000-0003-2042-4655}}}
\date{The Hong Kong Polytechnic University\\[2.5ex]\today}
% \date{\small\today}

\begin{document}

% \setlength{\tabcolsep}{2pt} % Change table padding

\maketitle



\begin{abstract}

\end{abstract}

\noindent{\small\textbf{Keywords --- } 1, 2, 3}

\section{Introduction}

This paper presents an overview of traveling food names and gastronomical terminology in East Asia. Some of the words we use in everyday life are richly embedded with contextual information on cultural contact and linguistic exchange. By eating around the Sinoshpere with open ears, we can discover unlikely connections between faraway peoples and cultures. Moreover, with the exploration of such words, we can learn about historical events that shaped our eating habits and changed our words. In this paper, we will look at words borrowed into and from Sinitic languages, focusing on the topic of food.

\section{Items one by one}

\subsection{Siok3 phang51}

Siok3 phang51 is a Japanese loanword in Taiwanese Hokkien. According to the definition in the \textit{Dictionary of Frequently-Used Taiwan Minnan} \parencite{ministry_of_education__2011}, phang51 itself is a loanword from Japanese, meaning 麵包‘bread’. Siok3 phang51 is derived from Japanese word 食パン, meaning 白麵包 ‘loaf bread’、吐司 ‘toast’. It is believed that Latin has influence the development of many European languages, such as Italian, French, Spanish. Bread was written as panis in Latin, pane in Italian, pain in French, pan in Spanish and pão in Portuguese. Not only do they have similar written forms but they share similar pronunciations. We should also note that the bread was written in「パン」pan in Japanese. How did the word travel from Europe to Japan?
The word for bread pan in Japanese has its roots in the period of sixteenth century, as pan comes from the Portuguese term for bread, pão. Tracing back to the history of the Age of Discovery, Portuguese was the first European arriving in Japan in 1543 \parencite{peters_bread_2017}. Since then, Portuguese and Spanish missionaries introduced Christianity to the country in 1549, when Jesuit Saint Francis Xavier established Japan’s first mission at Kagoshima. Christian missionaries played a role in spreading bread consumption from 1549 until the early 1600s through the ritual of the Eucharist \parencite{sheng_forging_2017}.
The Japanese word loaf bread食パンshokupan seems to be a combination of indigenous noun of eat, shoku and loanword from Portuguese bread, pan.
食+パン
shoku+pan
eat+bread, loaf bread
Phang51 in Taiwan Minnan dialect means 胖 fat.  

\subsection{Coriander}

Coriander (\textit{Coriandrum sativum} L.), also known as cilantro and Chinese parsley, is an annual herb originating in Western Asia. The fresh leaves are used as a culinary herb popular in Asian and Latin American cooking, especially in Indian, Middle Eastern, and Mexican cuisines. The dried seeds are used as a spice, and they are used in Europe more than the greens, with the exception of Portugal, where the fresh leaves are used generously \parencite{davidson_oxford_2014}. We can talk about essentially two products from one plant, valued for their different culinary properties. The leaves are called cilantro in the Americas, but also referred to as fresh coriander and coriander greens. Coriander seeds (often just termed coriander) are in fact the plant’s fruits, and ground coriander is a major component of Indian curry powders and pickles. While nowadays we consider coriander to be a culinary herb and spice, its historical role was more medicinal, and it was even used in perfume making.

It is one of the earliest Old World pants used as a condiment \parencite{zohary_domestication_2012}. The exact origin of this ancient crop is hard to define, but the native range is usually set somewhere between the East Mediterranean through the Transcaucasus to Pakistan. We do not know for sure when the species reached the Indian subcontinent \parencite{prance_cultural_2005}, where it enjoys widespread popularity. Coriander and its cultivars have been slowly spread and introduced to most places in Eurasia and grows almost globally in both wild and cultivated forms for thousands of years now.\footnote{For more details on the plant’s distribution, please refer to the Plants of the World Online (POWO) database at: \url{https://powo.science.kew.org/taxon/840760-1\#distribution-map}. Retrieved March 10, 2022.} \textit{Sativum} (Latin for `sown, planted') in the binomial name is a good indicator of this as well.

Coriander and its use have been documented in antiquity, but linguistic and archaeological evidence points to an even longer history. \textcite[163]{zohary_domestication_2012} gives a detailed list of archeobotanical findings, the oldest one of which dates to roughly eight thousand years ago, found in what is today Israel and Palestine. Coriander was also found in Egypt; remnants of desiccated coriander mericarps in Tutankhamun’s tomb (died 1323 BC) are proof for trade or cultivation by the ancient Egyptians \parencite{zohary_domestication_2012}. We know from the Ebers Papyrus – written around 1550 BC, considered amongst the oldest extant medical documents in the world – that the Egyptians used coriander in their medicinal practices \parencite{prance_cultural_2005}. An herbal remedy for headache from these ancient papyri is as follows:

\begin{quote}
\textsc{``Another remedy which the Goddess Isis prepared for the God Ra to drive out the pains that are in his head!}

\smallskip
Berry-of-the-Coriander --- I\\
Berry-of-the-Poppy-plant --- I\\
Wormwood --- I\\
Berry-of-the-sames-plant --- I\\
Berry-of-the-Juniper-plant --- I\\
Honey --- I
\smallskip

Make into one, mix with Honey, and smear therewith in order to make him well forthwith. When this remedy is used by him against all illnesses in the head and all sufferings and evils of any sort, he will instantly become well.'' \textcite[40]{bryan_papyrus_1930}
\end{quote}

\noindent Coriander was familiar to the bronze age Greeks as well, recorded on Linear B tablets from as early as 2000 BC. According to \textcite{chadwick_mycenaean_1976}, coriander -- reconstructed as \textit{koriadnon} -- must have been grown in both Mycenae and Knossos (on Crete) in considerable amounts, since the epigraphs refer to vast quantities. A tablet found in Pylos for example mentions 576 liters of coriander given to a perfume-maker. Later in antiquity, ancient Greek physicians Hippocrates and Dioscorides mentioned the medicinal properties of coriander, around 400 BC and 65 AD, respectively. Coriander was introduced to England by the Romans, and in 812 Charlemagne ordered its subjects to grow it on the farms of the Holy Roman Empire \parencite{prance_cultural_2005}. It was also one of the first herbs to be introduced to the American colonies, by 1670 it was found in Massachusetts. Due to their relative abundance and widespread distribution, coriander seeds never became a pricey commodity during the spice trade, and as an herb the value is in the freshness of the leaves: it is best when used locally. Nevertheless, the coriander seed world trade total value was at US\$192 million in 2019.\footnote{Data from OEC \url{https://oec.world/en/profile/hs92/coriander-seeds}. Retrieved March 10, 2022.}

\subsubsection{Names and etymology}

The word \textit{coriander} entered English in the 14th century via Old French \textit{coriandre}, from Latin \textit{coriandrum}. The Latin word is a borrowing from Greek \textit{koriannon}, a variant of \textit{koriandron}, of uncertain etymology. A shortened version of \textit{korion} also exists, among others. It is often repeated in both popular and scientific literature that the name of coriander comes from the Greek word \textit{koris} `bug, bedbug' (\textit{Cimex lectularius}), for its strong smell of the unripe fruit or the foliage \parencite[cf.][]{harper_coriander_nodate}. According to most contemporary writers, this idea is first promoted by Pliny the Elder (AD 23/24–79), some even purporting that Pliny named the plant after the stinking bug \parencite{oconnell_book_2016, cumo_encyclopedia_2013, prance_cultural_2005} (((O’Connell, 2016, p. 87; Cumo, 2013, p. 318-319; Prance \& Nesbitt, 2005, p. 152))). Pliny mentions coriander multiple times in remedies, stating that the best quality comes from Egypt. However, his monumental work, the \textit{Natural History} \parencite{pliny_the_elder_natural_1855} does not contain any statement of referring to bugs or the name, regarding coriander. The bedbug connection is not often discussed by lexicographers, and dictionaries generally do not endorse it (except \textit{Merriem-Webster’s Unabridged}). It seems to be a sort of false etymologization, taken for granted and copied for centuries, reaching as far as a statement in \textit{Encyclopaedia Britannica} (1911), although not without reason.

Coriander contains the same aldehydes a stink bug (\textit{Halyomorpha halys}) does, in fact the odor of the stink bug is often likened to that of coriander and (McGee, 2010), and a bed bug infestation reportedly has a similar olfactory experience \parencite{davidson_oxford_2014}. One of the reasons for coriander’s divisive nature is that not all of us perceive these chemicals the same way. Opposing opinions on the smell and benefits of it go back to the Middle Ages, arguments or giants, such as Galen (129--216) and Ibn Sina (c. 980--1037) for and against it were reported hundreds of years later \parencite[cf.][]{parkinson_theatrum_1640}. The hatred is so strong from some people, that a series of websites and social media groups are dedicated to it today, under slogans to the effect of “I HATE CILANTRO!” — plus, the term \emph{cilantrophobe} exists. All this has to do with our predisposed genetic differences in perceiving its sensory qualities, and some sources report on Europeans’ (i.e. non-traditional coriander consumers) strong aversive reactions \parencite{eriksson_genetic_2012}. It is also to be noted that the fruits of coriander slightly resemble the appearance of a bedbug as well, and we can speculate that this has also strengthened the connection. Anthropologist \textcite{leach_rehabilitating_2001} retraced the hateful remarks to French and English garden manuals from around the 1600s.
Consulting the Google Books Ngram Viewer \parencite{michel_quantitative_2011}, the first claim of coriander’s name deriving from koris in English is from 1640, in John Parkinson’s paramount \textit{Theatrum Botanicum} \parencite[918-919]{parkinson_theatrum_1640}. Other botanists before him also rushed to mention coriander’s stinking character, although not connecting it with the derivation of its name. These include John Gerard in his 1597 \textit{The Herball, or Generall Historie of Plantes} (p. 859) – who mostly plagiarised the Flemish father of botany, Rembert Dodoens and his 1586 work \textit{A nievve herball, or, Historie of plantes} (p. 313), calling it a “very stinking herbe”. In the following centuries both British and American botanists perpetuated this putative word origin. The Mycenaean Greek forms written in Linear B are recorded as ko-ri-ja-do-no /koriʰadnon/ and ko-ri-a2-da-na /korihadna/. \textcite[754]{beekes_etymological_2010} suggest a pre-Greek origin, citing the -dn- cluster, and dismisses both the folk etymologization by \textcite{frisk_griechisches_2021} Frisk (1960/2021) and a possible connection to Akkadian huri’ānu ‘coriander’ proposed by \textcite{szemerenyi_review_1971}. It is more likely then, that those who already disliked coriander (in this case, 16--17\textsuperscript{th}-century European botanists) played on the koris-korion link, and arbitrarily conformed the explanation of its name for some personal justification.
Coriander’s synonym, cilantro, is a doublet of coriander. It comes via Spanish culantro from the Late Latin coliandrum a variant of the classical coriandrum. The name cilantro entered English usage relatively late in 1907 \parencite{harper_coriander_nodate} and it usually refers to the fresh leaves people use as garnish. Its use is restricted to North and Latin America, mainly due to the herb’s rise in popularity with the emergence of Mexican cuisine in the U.S., bringing the word cilantro with it. (Culantro now is used for an indigenous Latin American herb (\textit{Eryngium foetidum}), also known as long coriander.)

As for the alternative name Chinese parsley, it refers to the leaves as well, never used to the seeds. Some authors and even authoritative encyclopedias presume that the name Chinese parsley is used in China/the Orient, which is a rather silly assumption, since geographic designations are not likely epithets to be used internally \parencite[cf.][]{davidson_oxford_2014, oconnell_book_2016}. The notion to relate coriander to some type of parsley is not far-fetched, the leaves look very similar to other herbs in the \textit{Apiaceae} family (parsley, dill, anise, fennel, etc.). Back to the term Chinese parsley, it apparently emerged in New York at the turn of the century. The phrase first appears in an 1832 U.S. court document, listing the items of a Chinese grocer, and later in an 1854 work titled \textit{The Transactions of the Royal Hawaiian Agricultural Society}, both without explanation. The first truly interesting story featuring Chinese parsley was an article in Frank Leslie's Popular Monthly, Volume 35 from 1893. The article, A celestial farm on Long Island, introduces how Chinese immigrants have set up farms in the Astoria neighborhood of Queens, in New York City \parencite{seitz_celestial_1893} . The article also contains illustrations, the prices for Chinese vegetables at the time, and identifies Chinese parsley as ``yen sai''\footnote{芫荽 \textit{yán sui}} in Chinese. Hence, it seems plausible that New Yorkers familiarized themselves with the herb from these Chinese farms and this is how the name gained some popularity.





\section{TODO}
\subsection{Tracy's work}
\begin{itemize}
    \item siok phang --- ALMOST DONE
    \item hodai
    \item lapsang souchong
\end{itemize}

\subsection{Gábor's work}
\begin{itemize}
    \item coriander --- ALMOST DONE
    \item cumin
    \item fennel
    \item Sichuan pepper
    \item star anise
\end{itemize}

\subsection{Full list of vocabulary to work on}
•	star anise (badiane)
•	castella
•	cha, chai, tea
•	coriander (cilantro, Chinese parsley)  
•	congee
•	cumin (zireh, ziran) 
•	fennel
•	hodai
•	ketchup
•	kopitiam
•	lapsang souchong
•	pekoe
•	sago
•	soy
•	Sichuan pepper (fagara)
•	siok phang
•	tempura
•	udon
•	wonton
•	bakmi
•	bakso
•	toko 土庫
•	kedai கடை
•	warung 亞朗
•	pasar بازار

% \section{Notes}

% How to organize this into a paper? 
% How to tell a research story? 
% Chronologically: by the time of the loan?

% How to group these items? 
% By area: SEA, EU, JP. 
% By source lang.: Minnan, Canto, Mandarin. 
% By the kind of item (spice, dish, fruit...) 

% What more similar items can we find? 

% Loanwords of Chinese origin in SEA or elsewhere (e.g. bakso bakmi from Hokkien) Food names that went into Chinese? 

% Group them by type of borrowing (phonetic or semantic...) transliteration, translation, calque?

%%%%%%%%%%%%%%%%%%%%


\printbibliography

\end{document}
