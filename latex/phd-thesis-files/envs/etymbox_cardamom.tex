\begin{etymology}\label{ety:cardamom}
\textbf{English} \textit{cardamom} `cardamom', (via post-classical Latin \textit{cardimomum}, a. 1398), ?ca. 1425
< later also from \textbf{Old French} \textit{cardemome} `cardamom', ca. 1170; cf. modern French \textit{cardamome}
< \textbf{Latin} \textit{cardamōmum} `cardamom', \nth{1} c. \AD{}
< \textbf{Hellenistic Greek} {καρδάμωμον} \textit{kardámōmon} `cardamom', haplological \gr{κάρδαμ-} \textit{kárdam-} `cress' + \gr{ἄμωμον} \textit{ámōmon} `an Indian spice plant', \nth{3} c. \BC{}
< \textbf{Ancient Greek} {κάρδαμον} \textit{kárdamon} `garden cress \taxon{Lepidium sativum}', prehaps a loanword (many plant names with \textit{-amon} are clear loanwords; the suffIx \textit{-amon} is known from Pre-Greek; ultimately of uncertain origin), \nth{4} c. \BC{}; cf. cognates classical Latin \textit{cardamum}
<\footnote{\textcite[s.v. cardamom]{oed}; \textcite[s.v. cardamome]{tlfi}; \textcite[s.v. cardamomum]{lewis_latin_1879}; \textcite[s.v. καρδάμωμον]{liddell_greek-english_1940}; \textcite[s.v. κάρδαμον]{liddell_greek-english_1940}; \textcite[644]{beekes_etymological_2010}}
\end{etymology}