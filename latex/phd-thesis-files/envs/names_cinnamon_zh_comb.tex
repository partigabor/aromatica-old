\begin{table}[!ht]
\centering
\begin{tabularx}{\textwidth}{@{}l>{\itshape \small}ll>{\itshape}lL>{\small}l@{}}
\toprule
\textbf{\#} & \multicolumn{1}{l}{\textbf{Species}} & \multicolumn{1}{l}{\textbf{Name}} & \multicolumn{1}{l}{\textbf{Tr.}} & \multicolumn{1}{l}{\textbf{Gloss}} & \multicolumn{1}{l}{\textbf{Source}} \\
\midrule
1	& Cinnamomum cassia	& \traditionalchinesefont{桂}	& guì	& cassia	& \textcite{defrancis_abc_2003} \\
2	& Cinnamomum cassia	& \traditionalchinesefont{桂皮}	& guìpí	& cassia-skin	& \textcite{defrancis_abc_2003} \\
3	& Cinnamomum cassia	& \traditionalchinesefont{桂心}	& guìxīn	& cassia-heart	& \textcite{hu_food_2005} \\
4	& Cinnamomum cassia	& \traditionalchinesefont{桂枝}	& guìzhī	& cassia-branches	& \textcite{hu_food_2005} \\
5	& Cinnamomum cassia	& \traditionalchinesefont{桂子}	& guìzǐ	& cassia-seeds	& \textcite{defrancis_abc_2003} \\
6	& Cinnamomum cassia	& \traditionalchinesefont{官桂}	& guānguì	& official-cassia	& \textcite{hu_food_2005} \\
\textbf{7}	& \textbf{Cinnamomum cassia}	& \textbf{\traditionalchinesefont{肉桂}}	& \textbf{ròuguì}	& \textbf{flesh-cassia}	& \textbf{\textcite{hu_food_2005}} \\
\textbf{8}	& \textbf{Cinnamomum verum}	& \textbf{\traditionalchinesefont{錫蘭肉桂}}	& \textbf{xīlán ròuguì}	& \textbf{Ceylon-flesh-cinnamon}	& \textbf{\textcite{wikipedia}} \\
\bottomrule
\end{tabularx}
\caption{Various names for cinnamon in Chinese.}
\label{table:names_cinnamon_zh}
\end{table}

