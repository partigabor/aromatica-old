\begin{etymology}\label{ety:geluzi}
\textbf{Mandarin Chinese} {葛縷子} \textit{gě​lǚ​zi} `caraway' [bean-hemp-seed?], phono-semantic matching; see \textit{shilo} `cumin and caraway'
< \textbf{Japanese} {葛縷子} \textit{karyuushi} `caraway', probably a transcription of Latin \textit{Carui}, or English \textit{caraway} + \textit{zi} (also キャラウェイ \textit{kyarawei} and 姫茴香 [princess-fennel-spice]), 1822
<\textss{?} from \textbf{English} \textit{caraway} `caraway', ca. 1440
 or from \textbf{Medieval Latin} \textit{carui} `caraway', or some allied Romanic form; cf. cognates French carvi, Italian carvi, Spanish carvi (whence Scots carvy, kervie), Old Spanish alcaravea, alcarahueya, Portuguese alcaravia, alcorovia
< \textbf{Arabic} {كراويا} \textit{karāwiyā} `caraway', (loaned to some Eurropean languages with \textit{al-} definite article; via Andalusian Arabic)
< \textbf{Aramaic} {\he{כַרְוָיָא}/\sy{ܟܲܪܘܵܝܵܐ}} \textit{karwāyā} `caraway'
< \textbf{Ancient Greek} {καρώ} \textit{karṓ} `caraway', a form of the word \textit{káron}, derived from \textit{káre} `head'; -ṓ form seems Pre-Greek (these forms could not immediately give the Arabic, hence possibly via *καρυΐα \textit{*karuḯa} a typical plant derivation form of καρώ \textit{karṓ}, κάρον \textit{káron}); cf. cognates Latin \textit{carum, careum}\footnote{\textcite[100]{kleeman_oxford_2010}; \textcite[s.v. caraway]{oed}; \textcite[s.v. caraway]{ahd}; \textcite[74]{corriente_dictionary_2008}; \textcites[207]{low_aramaeische_1881}[437-438]{low_flora_1924}; \textcites[653]{beekes_etymological_2010}[599]{sokoloff_dictionary_2002}}
\end{etymology}