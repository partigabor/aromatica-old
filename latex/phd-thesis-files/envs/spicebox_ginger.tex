\begin{spice}\label{spice:ginger}
\textsc{Ginger} \hfill \href{https://powo.science.kew.org/taxon/798372-1}{POWO} \\
\textbf{English:} \textit{ginger}. 
\textbf{Arabic:} {\arabicfont{زنجبيل}} \textit{zanjabīl}. 
\textbf{Chinese:} {\traditionalchinesefont{薑}} \textit{jiāng}. 
\textbf{Hungarian:} \textit{gyömbér}.  \\
\noindent{\color{black}\rule[0.5ex]{\linewidth}{.5pt}}
\begin{tabular}{@{}p{0.25\linewidth}@{}p{0.75\linewidth}@{}}
Plant species: & \taxonn{Zingiber officinale}{Roscoe} \\
Family: & \textit{Zingiberaceae} \\
part used: & rhizome \\
Region of origin: & South East Asia; India (secondary) \\
Cultivated in: & India; Jamaica; Nigeria; Sierra Leone \\
Color: & light yellow when fresh, beige when powdered \\
\end{tabular}
\end{spice}