\begin{etymology}\label{ety:zanjabil}
\textbf{Arabic} {زنجبيل} \textit{zanjabīl} `ginger', 609-632
< \textbf{Classical Syriac} {ܙܢܓܒܝܠ} \textit{zangabīl} `ginger'
< \textbf{Pahlavi} \textit{singibēr} `ginger', or via another Middle Iranian language
< \textbf{Sauraseni Prakrit} {𑀲𑀺𑀁𑀕𑀺𑀯𑁂𑀭} \textit{siṃgivera} `ginger'
<\textss{?} \textbf{Sanskrit} {शृङ्गवेर} \textit{śṛṅgavera} `ginger'
< \textbf{Dravidian} \textit{*cinki-wēr} `ginger', South dravidian nominal compound  from the etyma of Tamil and Malayalam \textit{iñci} (both with regular loss of an initial sibilant) + \textit{vēr} (Proto-Dravidian \textit{wēr}); the base of \textit{*cinki} is a loanword
< \textbf{unknown language} \textit{?} `ginger', unidentified Southeast Asian language; cf. cognates Khasi \textit{sying} /sʔiŋ/, Thai \textit{khing}, Vietnamese \textit{gừng}, Chinese \textit{jiāng}
<\textss{?} \textbf{Proto-Sino-Tibetan} \textit{*kjaŋ} `ginger'\footnote{\textcite{cal}; \textcite[90]{ciancaglini_iranian_2008}; \textcite[5]{krishnamurti_dravidian_2003}; \textcite{oed}}
\end{etymology}