\begin{table}[!ht]
\centering
\begin{tabularx}{\textwidth}{@{}l>{\itshape \small}ll>{\itshape}lL>{\small}l@{}}
\toprule
\textbf{\#} & \multicolumn{1}{l}{\textbf{Species}} & \multicolumn{1}{l}{\textbf{Name}} & \multicolumn{1}{l}{\textbf{Tr.}} & \multicolumn{1}{l}{\textbf{Gloss}} & \multicolumn{1}{l}{\textbf{Source}} \\
\midrule
\textbf{1}	& \textbf{Pimenta dioica}	& \textbf{\traditionalchinesefont{多香果}}	& \textbf{duōxiāngguǒ}	& \textbf{many-spice-fruit}	& \textbf{\textcite{kleeman_oxford_2010}} \\
2	& Pimenta dioica	& \traditionalchinesefont{全香子}	& quánxiāngzǐ	& all-spice-seed	& \textcite{spices_journey_quanxiangzi_2022} \\
3	& Pimenta dioica	& \traditionalchinesefont{甜胡椒}	& tiánhújiāo	& sweet-barbarian-pepper	& \textcite{yellowbridge} \\
4	& Pimenta dioica	& \traditionalchinesefont{牙買加胡椒}	& yámǎijiā hújiāo	& Jamaica-barbarian-pepper	& \textcite{mdbg} \\
5	& Pimenta dioica	& \traditionalchinesefont{眾香子}	& zhòngxiāngzǐ	& many-spice-seed	& \textcite{mdbg} \\
\bottomrule
\end{tabularx}
\caption{Various names for allspice in Chinese.}
\label{table:names_allspice_zh}
\end{table}

