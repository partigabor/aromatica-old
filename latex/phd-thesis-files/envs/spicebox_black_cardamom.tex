\begin{spice}\label{spice:black cardamom}
\textsc{Black cardamom} \hfill \href{https://powo.science.kew.org/taxon/urn:lsid:ipni.org:names:872166-1}{POWO} \\
\textbf{English:} \textit{black cardamom}; \textit{brown cardamom; greater cardamom; Indian cardamom; Nepal cardamom; Indian black cardamom; Bengal cardamom; big cardamom; hill cardamon; winged cardamom; fake cardamom; false cardamom; amomum*}. 
\textbf{Arabic:} {\arabicfont{قاقلة}} \textit{qāqulla}. 
\textbf{Chinese:} {\traditionalchinesefont{香豆蔻}} \textit{xiāngdòukòu} [fragrant-cardamom]; 嘎哥拉 \textit{gāgēlā}. 
\textbf{Hungarian:} \textit{fekete kardamom} [black cardamom].  \\
\noindent{\color{black}\rule[0.5ex]{\linewidth}{.5pt}}
\begin{tabular}{@{}p{0.25\linewidth}@{}p{0.75\linewidth}@{}}
Plant species: & \taxonn{Amomum subulatum}{Roxb.} \\
Family: & \textit{Zingiberaceae} \\
Plant part used: & fruit & seed \\
Region of origin: & Nepal to Central China \\
Cultivated in: & Himalayas \\
Color: & dark brown \\
\end{tabular}
\end{spice}