\subsection{Camphor}

% English,camphor
% Old French,camphore
% Medieval Latin,camphora
% Spanish,alcanfor
% Arabic,kāfūr
% Sanskrit,karpūra
% Old Malay,kapur


% EE:
% XV. Early forms are various, both disyll. and trisyll., camphire prevailing from XV to c.1800. — OF. camphore (later and mod. camphre) or medL. camphora — (prob. through Sp. alcanfor) Arab. kāfūr, ult. — Skr. karpūra-.

% OE:
% whitish, translucent, volatile substance with a penetrating odor, the product of trees in east Asia and Indonesia, extensively used in medicine, early 14c., caumfre, from Old French camphre, from Medieval Latin camfora, from Arabic kafur, perhaps via Sanskrit karpuram, from Malay (Austronesian) kapur "camphor tree." Related: Camphorated.

% MW:
% alteration (influenced by Medieval Latin & New Latin camphora) of Middle English caumfre, from Anglo-French, from Medieval Latin camphora, from Arabic kāfūr, from Malay kāpūr, probably of Austroasiatic origin (whence Sanskrit karpūra camphor); akin to Khmer kāpōr camphor

% First Known Use: 14th century (sense 1)


% AH:
% [Middle English caumfre, from Anglo-Norman, from Medieval Latin camphora, from Arabic kāfūr, possibly from Malay kapur; akin to Sanskrit karpūraḥ.]

% WK:
% From Old French camphore or Medieval Latin camphora, from Arabic كَافُور (kāfūr), via an Austronesian language such as Malay kapur from Sanskrit कर्पूर (karpūra). It is possible that the Sanskrit term was borrowed from Malay rather than vice versa. 


