\subsection{Frankincense}

% English,frankincense
% Old French,franc encens
% Mediavel Latin,francus incensus



% EE:
% frankincense 
% XIV. — OF. franc encens; see FRANK (‘of superior quality’), INCENSE.

% frank 

% †free XIII; bounteous, generous; †of superior quality XV; ingenuous, candid XVI. — (O)F. franc :- medL. francus free, indentical with the ethnic name (see FRANK), which acquired the sense ‘free’ because in Frankish Gaul full freedom was possessed only by those belonging to or adopted by the dominant people.

% Hence (from the sense †‘free of charge’ of the adj.) vb. superscribe (a letter, etc.) with one's signature to ensure free conveyance, (hence) stamp XVIII; facilitate the passage of XIX.

% incense1 

% aromatic gum burnt to produce a sweet smell XIII; smoke of this XIV. ME. ansens, encens — (O)F. encens — ecclL. incensum, sb. use of n. of incensus, pp. of incendere set fire to, f. IN-1 + *candere cause to glow (candēre glow).

% Hence vb. XIV.

% OE:
% frankincense (n.)

% aromatic gum resin from a certain type of tree, used anciently as incense and in religious rituals, late 14c., apparently from Old French franc encense, from franc "noble, true" (see frank (adj.)), in this case probably signifying "pure" or "of the highest quality," + encens "incense" (see incense (n.)).
% Entries linking to frankincense
% frank (adj.)

% c. 1300, "free, liberal, generous;" 1540s, "outspoken," from Old French franc "free (not servile); without hindrance, exempt from; sincere, genuine, open, gracious, generous; worthy, noble, illustrious" (12c.), from Medieval Latin francus "free, at liberty, exempt from service," as a noun, "a freeman, a Frank" (see Frank).

%     Frank, literally, free; the freedom may be in regard to one's own opinions, which is the same as openness, or in regard to things belonging to others, where the freedom may go so far as to be unpleasant, or it may disregard conventional ideas as to reticence. Hence, while openness is consistent with timidity, frankness implies some degree of boldness. [Century Dictionary]

% A generalization of the tribal name; the connection is that Franks, as the conquering class, alone had the status of freemen in a world that knew only free, captive, or slave. For sense connection of "being one of the nation" and "free," compare Latin liber "free," from the same root as German Leute "nation, people" (see liberal (adj.)) and Slavic "free" words (Old Church Slavonic svobodi, Polish swobodny, Serbo-Croatian slobodan) which are cognates of the first element in English sibling "brother, sister" (in Old English used more generally: "relative, kinsman"). For the later sense development, compare ingenuity.
% incense (n.)
% late 13c., "gum or other substance producing a sweet smell when burned," from Old French encens (12c.), from Late Latin incensum "burnt incense," literally "that which is burnt," noun use of neuter past participle of Latin incendere "set on fire" (see incendiary). Meaning "smoke or perfume of incense" is from late 14c.

% MW:
% Middle English fraunk encens, from fraunk, frank free, pure + encens incense — more at frank, incense
% First Known Use: 14th century (sense 1)



% Middle English, from Old French franc, from Medieval Latin francus, from Late Latin Francus, noun, Frank
% First Known Use: 14th century (sense 1)



% Middle English encens, incense, from Old French encens, from Late Latin incensum incense, from Latin, neuter of incensus, past participle of incendere to kindle, set on fire, irritate, from in- 2in- + -cendere to burn (akin to Latin candēre to shine, be glowing hot, be white) — more at candid
% First Known Use: 13th century (sense 1)


% AH:
% [Middle English frank encens, from Old French franc encens : franc, free, pure; see FRANK1 + encens, incense; see INCENSE2.]
% [Middle English, free, from Old French franc, from Late Latin Francus, Frank; see FRANK.]
% [Middle English encens, from Old French, from Latin incēnsum, from neuter past participle of incendere, to set on fire; see kand- in the Appendix of Indo-European roots.]

% WK:
% From Old French franc encens (“noble incense”). 
% From Latin Francus, thought to be from Frankish. 
% Borrowed from Late Latin incēnsum, from Latin incendō. 
