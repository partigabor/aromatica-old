\section{Ambergris}
\label{sec:ambergris}

\input{envs/spicebox_ambergris}




% EE:
% ambergris 
% wax-like substance found floating in tropical seas, and in the intestines of the sperm whale. XV. — (O)F. ambre gris ‘grey amber’; this is the orig. sense of amber (cf. preco), the word gris being added to distinguish the cetaceous secretion.

% amber
% †ambergris; yellow fossil resin. XIV. ME. aumbre — (O)F. ambre (medL. ambar(e), ambrum) — Arab. 'anbar ambergris, amber.

% OE:
% ambergris (n.)
% early 15c., from Old French ambre gris "gray amber," "a wax-like substance of ashy colour, found floating in tropical seas, a morbid secretion from the intestines of the sperm-whale. Used in perfumery, and formerly in cookery" [OED], via Medieval Latin from Arabic 'anbar (see amber). Its origin was known to Constantinus Africanus (obit c. 1087), but it was a mystery in Johnson's day, and he records nine different theories; "What sort of thing is Ambergrease?" was a type of a puzzling question beyond conjecture. King Charles II's favorite dish was said to be eggs and ambergris [Macauley, "History of England"].
% Praise is like ambergris; a little whiff of it, by snatches, is very agreeable; but when a man holds a whole lump of it to his nose, it is a stink and strikes you down. [Pope, c. 1720]
% French gris is from Frankish *gris or some other Germanic source (cognates: Dutch grijs, Old High German gris; see gray (adj.)).
% Entries linking to ambergris

% amber (n.)
% mid-14c., ambre grice "ambergris; perfume made from ambergris," from the phrase in Old French (13c.) and Medieval Latin, from Arabic 'anbar "ambergris, morbid secretion of sperm-whale intestines used in perfumes and cookery" (see ambergris), which was introduced in the West at the time of the Crusades. Arabic -nb- often is pronounced "-mb-."

% In Europe, amber was extended to fossil resins from the Baltic (late 13c. in Anglo-Latin; c. 1400 in English), and this has become the main sense as the use of ambergris has waned. Perhaps the perceived connection is that both were found washed up on seashores. Or perhaps it is a different word entirely, of unknown origin. Formerly they were distinguished as white or yellow amber for the Baltic fossil resin and ambergris "gray amber;" French distinguished the two substances as ambre gris and ambre jaune.

% Remarkable for its static electricity properties, Baltic amber was known to the Romans as electrum (compare electric). Amber as an adjective in English is from c. 1500; as a color name 1735. In the Old Testament it translates Hebrew chashmal, a shining metal.
% gray (adj.)
% "of a color between white and black; having little or no color or luminosity," Old English græg "gray" (Mercian grei), from Proto-Germanic *grewa- "gray" (source also of Old Norse grar, Old Frisian gre, Middle Dutch gra, Dutch graw, Old High German grao, German grau), with no certain connections outside Germanic. French gris, Spanish gris, Italian grigio, Medieval Latin griseus are Germanic loan-words. The spelling distinction between British grey and US gray developed 20c. Expression the gray mare is the better horse in reference to households ruled by wives is recorded from 1540s.

% MW:
% ambergris from Middle English ambregris, from Middle French ambre gris, from ambre ambergris, amber + gris gray; ambergrease by folk etymology from ambergris — more at amber, grizzle
% First Known Use: 15th century

% amber
% Middle English ambra, ambre, from Middle French \& Medieval Latin; Middle French ambre, from Medieval Latin ambra, ambar, from Arabic ʽanbar ambergris
% First Known Use: 15th century (sense 1)

% gris
% Middle English grisel, from Middle French, from Old French, from gris gray, of Germanic origin; akin to Old Frisian, Old Saxon, \& Old High German grīs gray, Old Norse grīss pig, and perhaps to Old English grǣg gray — more at gray

% gray
% Middle English, from Old English grǣg; akin to Old High German grāo gray, Old Norse grār, Old Slavic zĭrĕti to see, look
% First Known Use: before 12th century (sense 1a)



% AH:
% [Middle English, from Old French ambre gris : ambre, amber; see AMBER + gris, gray; see GRISAILLE.]
% [Middle English ambre, from Old French, from Medieval Latin ambra, ambar, from Arabic 'anbar, ambergris, amber.]
% [French, from gris, gray, from Old French, from Frankish *grīs.]

% WK:
% Old French ambre gris (“grey amber”). Though the term was initially spaced as two words, single-word forms predominated by the 19th century. In the 17th century, folk etymologies interpreting the term as amber grease or amber [of] Greece enjoyed some popularity.[1] 

% From Middle French ambre, from Arabic عَنْبَر (ʿanbar, “ambergris”), from Middle Persian ʾmbl (ambar, “ambergris”). 

% From Old French or Old Occitan, both from Frankish *grīs, from Proto-Germanic *grīsaz (“grey”). Akin to Old High German grīs (“grey”) (German greis) and Dutch grijs (“grey”). More at grizzle. 
% From Middle English grisel, gryselle, from Old French grisel, from gris (“grey”), from Frankish *grīs, from Proto-Germanic *grīsaz. 
