\documentclass[12pt]{article}
\usepackage[a4paper, margin=3cm]{geometry}
\usepackage{graphicx}
\usepackage{wrapfig}
\usepackage{csquotes}
\usepackage{booktabs}
\usepackage{soul} % for strikethrough \st{} and highlight \hl{}
\usepackage[dvipsnames]{xcolor} % Color names
\usepackage{hyperref}
\hypersetup{
    colorlinks=true,
    linkcolor=OliveGreen,
    anchorcolor=black,
    citecolor=OliveGreen,
    filecolor=black,
    menucolor=black,
    runcolor=black,
    urlcolor=OliveGreen}
\hypersetup{breaklinks=true}
\usepackage{url}
\usepackage{authblk}
\usepackage[final]{microtype}



% Bibliography information processing
\usepackage[
    backend=biber,
    style=apa,
    autocite=inline,
    sorting=nyt,
    sortcites=true,
    backref=false,
    backrefstyle=three,
    abbreviate=true,
    block=space,
]{biblatex}
% \DefineBibliographyStrings{american}{
%     backrefpage={Cited on p.},
%     backrefpages={Cited on pp.}
% }
\apptocmd{\sloppy}{\hbadness 10000\relax}{}{}
\appto{\bibsetup}{\sloppy}
\DeclareFieldFormat{apacase}{#1} % Title casing in APA
% \AtBeginBibliography{\small} % Can make the font in bibliography smaller

\addbibresource{../static/bibliography/parti.bib} %



% Languages, scripts, and fonts

% Font packages
\usepackage{libertine} % Latin, Greek, Cyrillic, Hebrew

% Fontspec
\usepackage{fontspec}
\defaultfontfeatures{Ligatures={TeX}}

% % Language settings
\usepackage[main=english, bidi=basic]{babel}

% Automatic typesetting of certain writing systems
\babelprovide[import=ar, onchar=ids fonts]{arabic}
% \babelprovide[import=he, onchar=ids fonts]{hebrew}
\babelprovide[import=sa-Deva, onchar=ids fonts]{sanskrit-devanagari}

% Define fonts (https://tug.org/FontCatalogue/)
% \babelfont{rm}{Linux Libertine}
% \babelfont{sf}{Linux Biolinum}
% \babelfont{tt}{Inconsolata}

\babelfont[*arabic]{rm}{Noto Naskh Arabic} % Amiri
% \babelfont[*hebrew]{rm}{Linux Libertine} % Noto Serif Hebrew
\babelfont[*devanagari]{rm}[Renderer=Harfbuzz]{Noto Serif Devanagari} 
% % Settigs: Scale=MatchUppercase; Scale=MatchLowercase; Scale=1.0; Language=Default

% East Asian scripts
% \usepackage{kotex} % Support for KR, JP, some TC (not SC).
% \setmainhangulfont{Noto Serif CJK KR} % Only on Overleaf
% \setmainhangulfont[
    % Path=./fonts/,
    % Extension = .otf,
    % UprightFont=*-Medium,
    % BoldFont=*-Bold
    % ]{NotoSerifKR}

\newfontfamily{\traditionalchinesefont}[
    Path=./fonts/,
    Extension = .otf,
    UprightFont=*-Medium,
    BoldFont=*-Bold
    ]{NotoSerifTC}

% \newfontfamily{\simplifiedchinesefont}[
%     Path=./fonts/,
%     Extension = .otf,
%     UprightFont=*-Medium,
%     BoldFont=*-Bold,
%     ]{NotoSerifSC}

% Other scripts
\newfontfamily{\tibetanfont}[Script=Tibetan]{Noto Serif Tibetan}

% Commands to use scripts
\newcommand{\bo}[1]{\tibetanfont{#1}\rmfamily}
\newcommand{\tc}[1]{\traditionalchinesefont{#1}\rmfamily}
\newcommand{\zh}[1]{\simplifiedchinesefont{#1}\rmfamily}

% Keywords command
\providecommand{\keywords}[1]{\small\noindent\textbf{\textit{Keywords ---}}#1}

% ORCiD
\definecolor{orcid}{HTML}{A6CE39}
\usepackage{academicons}
\newcommand{\orcid}[1]{\href{https://orcid.org/#1}{\textcolor{orcid}{\aiOrcid}}}



\title{Cardamoms, and a new etymology for Chinese \textit{dòukòu} `cardamom; nutmeg'}
\author[1]{{\small\orcid{0000-0003-2042-4655}}~Gábor Parti}
\author[2]{{\small\orcid{}}~Ian Joo}
\affil[1,2]{The Hong Kong Polytechnic University}
\affil[2]{Nagoya University of Commerce and Business}
\date{\small{February 2024}}

\begin{document}

\maketitle

\begin{abstract}
    This paper aims to provide a new etymological explanation for the Chinese word \tc{豆蔻} \textit{dòukòu} `cardamom; nutmeg' and explore the potential trade-language origins of the term. We propose that this word is a borrowing that may have entered the Chinese lexicon via Southern Min, originating from an Indian Ocean trade-language used on the ancient Maritime Silk Road during the Tang era (618--907). Our findings suggests that \textit{dòukòu} entered the Chinese lexicon alongside the economic products it denotes, and underwent phono-semantic matching that obscured its foreign origin. We present our rationale in four main points, detailing our observations regarding 1) the earliest written records, 2) character composition, 3) available lexicographical data, and 4) an overview of regional \textit{Wanderwörter} (wandering loanwords) as evidence supporting our hypothesis. After uncovering the lexicogenesis of this word, we discuss the linguistic and historical plausibility of our claims, and explore potential candidate etymons, such as Prakrit and Arabic. Additionally, we also clarify the situation surrounding cardamoms from an Asian perspective, focusing on their identities, origins, and terminology in Chinese.

\end{abstract}

\noindent{\small\textbf{\textit{Keywords --- }} Cardamom, \tc{豆蔻}, Etymology, Trade-language, Chinese, Southern Min, Indian Ocean, Spice Trade}

\section*{Notes and todo list}

\begin{itemize}
    \item Text in \hl{highlight} needs to be checked; corrected; finalized.
    \item ?? indicates need for explanation; verification; citation.
\end{itemize}

\clearpage

\section{Introduction}

Cardamom is a popular spice with a history spanning thousands of years of culinary and medicinal use; the reader might recall the vibrant aroma of the bright green pods. Less well-known is the fact however, that the name \textit{cardamom} can refer to multiple different kinds of fragrant fruits, besides the familiar seed pods of the ``true cardamom'', \textit{Elettaria cardamomum}. Furthermore, delving deeper into the intricacies of cardamoms from an Asian perspective reveals a complex array of words and materials, as many prototypical cardamoms grown in various regions (e.g., Himalayas, Indochina, Java, Yunnan) are botanically distinct species, and their names and identities have become entangled along the paths of diffusion.

In this study, we examine the situation in Chinese, where the corresponding word is \tc{豆蔻} \textit{dòukòu} -- also referring to another important spice, nutmeg. Similarly to its equivalent in English, \textit{dòukòu} acts as an umbrella term that can refer to any of a number of cardamom-type fruits and other closely associated spices, its precise meaning defined by regional and dialectal actualities, and narrowed by modifying \hl{adjectives}.

The main objective of this paper is to offer a novel etymological explanation for the Chinese term \tc{豆蔻} \textit{dòukòu} 'cardamom; nutmeg', and explore its potential trade-language origins. We posit that this term is a loanword that may have been incorporated into Chinese via Southern Min, originating from a yet unidentified Indian Ocean trade-language that was in use along the ancient Maritime Silk Road during the Tang dynasty (618--907), when this term first appears in written records. Our research suggests that this word entered the Chinese lexicon alongside the economic products it denotes, and underwent phono-semantic matching that masks its foreign origin. In Section \ref{sec:etymology}, we present our reasoning in four main points, outlining our observations on 1) the earliest written records, 2) character composition, 3) available lexicographical data, and 4) an overview of seemingly related regional \textit{Wanderwörter} (wandering loanwords) from the Indian Ocean world, as evidence supporting our hypothesis. Following the elucidation of the lexicogenesis of this term, we examine potential candidate etymons, such as Prakrit and Arabic, discuss the linguistic and historical plausibility of our claims, and acknowledge the limitations.

% Tamil?
% something about history?

% \textit{Dòukòu} first appears as \textit{báidòukòu} [white-cardamom] in a Tang-era miscellany called \textit{Youyang Zazu}, where it is reported to come from the land of \hl{Kakkola}, describing the white, round cardamoms of (\textit{Wurfbainia vera} or \textit{Wurfbainia compacta}) originally sourced either from Siam or Java. According to this 9th-century source, the spice is called \tc{多骨} \textit{duōgǔ}, and the perceived Middle Chinese pronunciation, /tɑ-kuət̚/ ??, makes possible for it to be a sinicized transcription of an \hl{Indo-Aryan} word, e.g., from Prakrit. On the other hand, the same does not apply for \tc{豆蔻} \textit{dòukòu}, whose Middle Chinese pronunciation would be /dəuH-həuH/ ??. 

% personal communication, August 17, 2022

% As an alternative, we propose that \textit{dòukòu} could be a loan via Southern Min. Based on Kwok's reconstruction of Proto-Southern-Min, the equivalent of \textit{dòukòu} would be /*tɑu-khɑu/ \parencite{kwok_2018_southern}, more similar to the Indic words in question. 

% This is historically plausible, given that Fujian was in direct contact with the maritime traders responsible for the bulk of the spice trade, and that Sanskrit was one of the main languages of the Srivijaya Empire encompassing Kakkola around this time. In this brief etymological study, we clear the air around cardamom nomenclature, and reveal a possible trace that Fujianese traders have left on modern Chinese. This is not only interesting in terms of food history, but also in terms of the linguistic history of Chinese, as few loanwords from Southern Min have made their way into Middle Chinese.

The paper also seeks to clarify the situation surrounding the multitude of cardamoms in Asia, focusing on their identities, origins, and terminologies within a Chinese context. As it was just mentioned above, the meaning of the word \textit{dòukòu} itself can be ambiguous, and has been so since its earliest emergence, as it can refer to several different aromatic fruits and seeds. Therefore, we begin with a brief introduction to what cardamoms are, identifying the relevant plants and products known in Asia, accompanying with their nomenclature in Chinese. This will establish a foundation for our linguistic inquiries that is based on geobotanical realities.

\section{Cardamom conondrum}\label{sec:one}

\subsection{What is cardamom?}

What people mean by the name \textit{cardamom} depends on who we ask, where we are, and in what language we are operating -- for the word \textit{cardamom} and its translations can denote multiple different spices. In all cases, this name will refer to an aromatic plant yielding a culinary and medicinal spice, and in all likelihood, most readers will think of the green, dried fruits seen in Figure \ref{fig:seedpods}.

\begin{figure}[!h]
    \centering
    \includegraphics[width=0.5\textwidth]{imgs/cardamom_seed_pods.png}
    \caption{The seed pods of cardamom (Photo by Karyna Panchenko on \href{https://unsplash.com/photos/a-pile-of-green-cardamoas-sitting-on-top-of-a-white-table-9_IoYA6EdpY}{Unsplash})}
    \label{fig:seedpods}
\end{figure}

% \begin{wrapfigure}{r}{0.5\textwidth}
%     \centering
%     \includegraphics[width=0.5\textwidth]{imgs/cardamom_seed_pods.png}
%     \caption{The seed pods of cardamom}
%     \label{fig:seedpods}
% \end{wrapfigure}

% Brian Arthur
% https://commons.wikimedia.org/wiki/File:BlackCardamom.jpg

\begin{figure}[!h]
    \centering
    \includegraphics[width=0.45\textwidth]{imgs/cardamom.png}
    \caption{\textit{E. cardamomum} in \textit{Köhler's Medizinal Pflanzen} \pvolcite[]{2}[186]{koehler_1887_koehler}}
    \label{fig:plant}
\end{figure}

% \begin{wrapfigure}{r}{0.5\textwidth}
%     \centering
%     \includegraphics[width=0.5\textwidth]{imgs/cardamom.png}
%     \caption{\textit{Elettaria cardamomum} in \textit{Köhler's Medizinal Pflanzen} \pvolcite[]{2}[186]{koehler_1887_koehler}}
%     \label{fig:cardamom}
% \end{wrapfigure}

In Europe, the Middle East, India, and many other parts of the world, \textit{cardamom} typically refers to the seed pods of \textit{Elettaria cardamomum} (L.) Maton,\footnote{\textit{Elettaria cardamomum} in Plants of the World Online \parencite{powo}: \url{https://powo.science.kew.org/taxon/796556-1}} which is an aromatic flowering plant in the ginger family (\textit{Zingiberaceae}) -- see Figure \ref{fig:plant}. It is native to India's Western Ghats, the mountain range streching along the Malabar coast of Kerala. This is the same region where black pepper hails from, and in an Indian context they are often called ``the king and queen of spices'' -- cardamom being the queen \parencite{nair_2020_geographya}. 

It is a well-known culinary spice with a long history of commerce, also used in the infusion of beverages and perfumery. Think of Indian cuisine, Arabic coffee, Egyptian perfumes, etc. Cardamom has been in use in India since the Vedic period and was traded to and through Mesopotamia since the Bronze Age \parencite{ravindran_2002_cardamom}. 

In Europe, cardamom has been known since antiquity and used as medicine. Greek physicians, such as Dioscorides and Galen, have documented its healing properties, in some cases even correctly identifying India as its place of origin \parencites{parry_1969_spices}{anderson_2023_history}. Historiographers Theophrastus and Pliny (the Elder) described two types of cardamom: the superior καρδάμωμον \textit{kardámōmon} (\textit{E. cardamomum}) and the inferior ἄμωμον \textit{ámōmon} we now mostly call ``black cardamom'' (\textit{Amomum subulatum}) \parencite{prance_2005_cultural}, the latter of which has since lost its prevalence in Europe. During the Middle Ages, cardamom was traded in the Mediterranean primarily by Arab merchants to Venice, a monopoly broken by the Portuguese in the 16\textsuperscript{th} century, who established direct trade with the Malabar coast \parencite{cumo_2013_encyclopedia}. In his 1596 \textit{Itinerario}, Dutchman Linschoten reported ``lesser'' and ``greater'' cardamoms being used in south India, corresponding to the smaller green cardamoms, and the larger black cardamoms mentioned above \parencite{nair_2006_agronomy}.

Cardamom is still valuable today, it is the 3\textsuperscript{rd} most expensive spice after saffron and vanilla, and it is also known as \textit{green cardamom} and \textit{true cardamom}. 

% Green cardamom is virtually unknown in China,

% In China, cardamoms are exotic or semi exotic, Chinese sources feature significant trade in cardamoms during the Song dynasty (960--1279) \parencite{prance_2005_cultural}.

\subsection{False cardamoms}

Now, the name \textit{true cardamom} implies the existence of ``false'' ones, which is not so unusual in English spice nomenclature; it can happen for two reasons. First, the term \textit{false} and its synonyms are typically used to designate alternative economic products (cf. \textit{false peppers}). Second, they hint at practices of adulteration (cf. \textit{bastard saffron}). As one of the strategies employed to distinguish spices is by using a marker of authenticity, words, such as \textit{true, real, false, fake, bastard} can occur in vernacular names. In our case, the former holds true: there are at least a dozen other spice-bearing aromatic plants in the ginger family referred to as \textit{cardamoms}, two of which we have already mentioned. Until very recently, almost all cardamom-like plants were neatly sorted into two genera, the genus \textit{Amomum} with a native range spanning from South to Southeast Asia, and the genus \textit{Aframomum} distributed in Africa. Today, many of them boast with new and revised scientific names, which can make their discussion quite overwhelming.\footnote{Modern taxonomic classification of plants started with \textcite{linnaeus_1758_systema}, but it is still developing to this day -- English botanist Burkill have already lamented on the ``unwelcome'' renaming of some cardamom species almost a hundred years ago \parencite[see][]{burkill_1930_cardamoms}.}

So, why are these other spices called \textit{cardamoms} as well? In short, these plants and their products are named \textit{cardamoms} because of their physical, chemical, and functional similarity to a prototype spice, especially in one distinct feature: they are all aromatic seed pods containing small seeds. The compact fruits' shared morphology is distinctivie enough to warrant association with an already known item, which influences the naming of newly acquired and encountered spices. In a western context, the prototype spice is the green cardamom we know (\textit{E. cardamomum}), hence the name \textit{true cardamom}. We take the idea of a ``prototype spice'' from the prototype theory in cognitive science, as this seems to illuminate best the categorization and naming practices of new spices during acculturation \Parencite[see][]{parti_2023_mapping}.

To clarify, false cardamoms are not fake, what is considered ``false'' is determined by the predominant prototype spice in a given culture (whether it is indigenous to the region, or the first of its kind encountered), and the perceived value of the spice affected by notions of quality and scarcity. For instance, true cinnamon from Ceylon has historically been regarded as superior to cassia or Chinese cinnamon, leading to the use of names such as \textit{true cinnamon} (even dictating the binomial name \textit{Cinnamomum verum}, meaning `true cinnamon' in Latin) and \textit{bastard cinnamon}.

\subsection{Prototype cardamoms in Asia}

While for many of the concept of cardamom is synonymous with the dried, green, camphorous seed pods, there are regions in Asia where the ``default'' cardamom looks and tastes a bit different. In the Eastern Himalayas (Nepal, Bhutan, Northeast India, Bengal) we find \textit{A. subulatum}, the smoky \textit{black cardamom} also known as  \textit{greater cardamom}, \textit{brown cardamom}, \textit{Nepal cardamom}, etc. In Yunnan, Southwestern China there is the even larger \textit{Lanxangia tsaoko}, known as \textit{Chinese black cardamom}, \textit{tsaoko cardamom}, or simply \textit{tsaoko}. Southeast Asia has two species that yield little, white, round cardamoms: \textit{Wurfbainia vera}\footnote{Formerly \textit{Amomum verum} Blackw., it is the first plant species to be named by a woman, Elizabeth Blackwell, in 1757.} mainly grown in the Cardamom Mountains of Cambodia, commonly known as \textit{round cardamom}, \textit{Siam cardamom}, or \textit{krervanh}, and \textit{Wurfbainia compacta}, growing in Indonesia known as \textit{Java round cardamom} and \textit{kepulaga}. These are few of the most important cardamom-like plants in Asia that define local or regional culinary traditions as prototypical cardamoms. Zoning in on our main goal we can pose the question: Which ones is the cardamom in China?

Studying the English vernacular spice names can be quite revealing, as they contain the name of the prototype spice, modified with an adjective of color, a place of origin, or both; occasionally using a native name.

The following is a list of common names collected from historical and botanical literature on spices 

\parencites{dalby_2000_dangerous}{prance_2005_cultural}

% {hill_2004_contemporary}{anderson_2023_history}.

\begin{quote}
    amomum, bastard cardamom, bastard Siamese cardamom, Bengal cardamom, big cardamom, black cardamom, brown cardamom, Cameroon cardamom, Chinese black cardamom, Chinese cardamom, Ethiopian cardamom, fake cardamom, false cardamom, greater cardamom, green cardamom, Guinea cardamom, hill cardamon, Indian black cardamom, Indian cardamom, Indonesian cardamom, Java cardamom, Java round cardamom, Java white cardamom, kepulaga, korarima, krervanh, large cardamom, Madagascar cardamom, Malabar cardamom, Nepal cardamom, round cardamom, round Chinese cardamom, Siam cardamom, Tavoy cardamom, Thai cardamom, true cardamom, tsao-ko cardamom, wild Siamese cardamom, winged cardamom, Yunnan cardamom
\end{quote}

Selected sources: , etc.  




---


% \begin{table}[ht]
%     \centering
%     \begin{tabular}{p{10cm}}
%     \toprule
%     Alpinia galanga (L.) Willd. \\
%     Alpinia globose (Lour.) Horan. syn. Amomum globosum Lour. \\
%     Alpinia hainanensis K.Schum. syn. Alpinia katsumadai Hayata \\
%     Amomum maximum Roxb. \\
%     Amomum subulatum Roxb. \\
%     Elettaria cardamomum (L.) Maton syn. Amomum cardamomum L. \\
%     Hornstedtia costata (Roxb.) K.Schum. syn. Amomum costatum (Roxb.) Benth. ex Baker \\
%     Lanxangia tsao-ko (Crevost \& Lemarié) M.F.Newman \& Skornick. syn. Amomum hongtsaoko Liang et Fang; A. tsao-ko Crevost et Lemaire \\
%     Wurfbainia aromatica (Roxb.) Škorničk. \& A.D.Poulsen syn. Amomum aromaticum Roxb. \\
%     Wurfbainia compacta (Sol. ex Maton) Škorničk. \& A.D.Poulsen syn. Amomum compactum Sol. ex Maton; Amomum kepulaga Sprague \& Burkill \\
%     Wurfbainia vera (Blackw.) Škorničk. \& A.D.Poulsen syn. Amomum krervanh Pierre \& Gagnep.; Amomum kravanh Pierre ex Gagnep. \\
%     Wurfbainia villosa (Lour.) Škorničk. \& A.D.Poulsen syn. Amomum villosum Lour. \\
%     Aframomum melegueta K.Schum. \\
%     Aframomum grana-paradisi (L.) K.Schum. \\
%     Aframomum cereum (Hook.f.) K.Schum. syn.Aframomum sceptrum (Oliv. \& D.Hanb.) K.Schum. \\
%     Aframomum exscapum (Sims) Hepper \\
%     Aframomum alboviolaceum (Ridl.) K.Schum. syn. Aframomum macrospermum (Sm.) Burkill \\
%     Aframomum angustifolium (Sonn.) K.Schum. \\
%     Aframomum corrorima (A.Braun) P.C.M.Jansen syn. Amomum korarima J.Pereira \\
%     Aframomum daniellii (Hook.f.) K.Schum. syn. Aframomum hanburyi K.Schum. \\
%     \bottomrule
%     \end{tabular}
%     \caption{List of Spice-bearing Aromatic Plants}
%     \label{tab:spice-plants}
%     \end{table}

All the relevant spice bearing plants…

Alpinia galanga (L.) Willd. 
Alpinia globose (Lour.) Horan. syn. Amomum globosum Lour.
Alpinia hainanensis K.Schum. syn. Alpinia katsumadai Hayata
Amomum maximum Roxb.	
Amomum subulatum Roxb.	
Elettaria cardamomum (L.) Maton syn. Amomum cardamomum L.
Hornstedtia costata (Roxb.) K.Schum. syn. Amomum costatum (Roxb.) Benth. ex Baker
Lanxangia tsao-ko (Crevost \& Lemarié) M.F.Newman \& Skornick. syn. Amomum hongtsaoko Liang et Fang; A. tsao-ko Crevost et Lemaire
Wurfbainia aromatica (Roxb.) Škorničk. \& A.D.Poulsen syn. Amomum aromaticum Roxb.
Wurfbainia compacta (Sol. ex Maton) Škorničk. \& A.D.Poulsen syn. Amomum compactum Sol. ex Maton; Amomum kepulaga Sprague \& Burkill
Wurfbainia vera (Blackw.) Škorničk. \& A.D.Poulsen syn. Amomum krervanh Pierre \& Gagnep.; Amomum kravanh Pierre ex Gagnep.
Wurfbainia villosa (Lour.) Škorničk. \& A.D.Poulsen syn. Amomum villosum Lour.
Aframomum melegueta K.Schum.	Aframomum grana-paradisi (L.) K.Schum. 
Aframomum cereum (Hook.f.) K.Schum. syn.Aframomum sceptrum (Oliv. \& D.Hanb.) K.Schum.
Aframomum exscapum (Sims) Hepper	
Aframomum alboviolaceum (Ridl.) K.Schum. syn. Aframomum macrospermum (Sm.) Burkill
Aframomum angustifolium (Sonn.) K.Schum.	
Aframomum corrorima (A.Braun) P.C.M.Jansen	 syn. Amomum korarima J.Pereira
Aframomum daniellii	(Hook.f.) K.Schum. syn. Aframomum hanburyi K.Schum.

Now, the whole range of plants from the cardamom-group is long and unnecessary, but now we have a question: Which one is doukou?



\begin{table}[ht]
    \centering
    \begin{tabular}{llllll}
    \toprule
    \textbf{Scientific Name} & \textbf{English Name} & \textbf{Chinese Name} & \textbf{Pinyin} & \textbf{Literal Translation} \\
    \midrule
    Elettaria cardamomum & cardamom & 小豆蔻 & xiǎodòukòu & little-bean-cardamom \\
    Elettaria cardamomum & cardamom & 綠豆蔻 & green-bean-cardamom \\
    Amomum subulatum & black cardamom & 香豆蔻 & xiāngdòukòu & fragrant-cardamom \\
    Lanxangia tsao-ko & tsaoko & 草果 & cǎoguǒ & herb-fruit \\
    Alpinia hainanensis & galangal seed & 草豆蔻 & cǎodòukòu & herb-cardamom \\
    Alpinia galanga & greater galangal & 红豆蔻 & hóngdòukòu & red-cardamom \\
    Wurfbainia compacta & kepulaga & 爪哇白豆蔻 & zhǎowā báidòukòu & Java-white-cardamom \\
    Wurfbainia vera & Siam cardamom & 白豆蔻 & báidòukòu & white-cardamom \\
    Wurfbainia villosa & wild Siamese cardamom & 砂仁 & shārén & granule-kernel \\
    \bottomrule
    \end{tabular}
    \caption{Names of Cardamom in Different Languages}
    \label{tab:cardamom-names}
    \end{table}


---

Well, the answer is: a few of these, and more. In a Chinese context we have our not so well-known green cardamom, black cardamom from the Himalayas, the two white cardamoms from Southeast Asia, some semi-exotic ones that look like tiny brains, and the reddish fruits of galangal.

Alpinia galanga (L.) Willd. 
Alpinia hainanensis K.Schum.	
Amomum subulatum Roxb.	
Elettaria cardamomum (L.) Maton 
Wurfbainia compacta (Sol. ex Maton) Škorničk. \& A.D.Poulsen
Wurfbainia vera (Blackw.) Škorničk. \& A.D.Poulsen

---

X cardamom  = X豆蔻 x dòukòu



Here is a table showing the same plant products with their Chinese names; consider the examples: red-cardamom, fragrant-cardamom, little-cardamom, white-cardamom, and so on.
You can see that Chinese doukou corresponds to cardamom; at least on a pragmatic level if not botanically. 
Dòukòu is a generic term and Chinese too also adds additional modifiers of color, place, or some other feature, with many-many dialectal variations that are out of scope for now, but basically we have a formula.
As bonus, we also have nutmeg, which is called ròudòukòu (flesh cardamom) in Chinese.

---

Nutmeg

It is a seed of a tree of an unrelated species from a different family, native to the Maluku islands of Indonesia, famously called the Spice Islands in colonial times, and until the 18th century it only grew here. It was one of the most prized products of the spice trade, for half a kilo you could buy a house in London. 


---

Origins

Map

South \& Southeast Asia
Biodiversity hotspot
Maritime Silk Road

By the way, these aromatic plants come from this region surrounding Mainland Southeast Asia, which is a biodiversity superhotspot, with many stops along the Maritime Silk Road.









\section{Etymological breadcrumbs}\label{sec:etymology}

Scholars have a reatively good understanding of the etymology of the English word \textit{cardamom}, which is usually reconstructed along the following lines:

\begin{quote}
    \textbf{English} \textit{cardamom} `cardamom' ca. 1425, via \textbf{post-classical Latin} \textit{cardimomum}, a. 1398
    < later also from \textbf{Old French} \textit{cardemome} `cardamom', ca. 1170; cf. modern French \textit{cardamome}
    < \textbf{Latin} \textit{cardamōmum} `cardamom', 1st c. AD
    < \textbf{Hellenistic Greek} {καρδάμωμον} \textit{kardámōmon} `cardamom', haplological κάρδαμ- \textit{kárdam-} `cress' + ἄμωμον \textit{ámōmon} `an Indian spice plant', 3rd c. BC
    < \textbf{Ancient Greek} {κάρδαμον} \textit{kárdamon} `garden cress, \textit{Lepidium sativum}', prehaps a loanword (many plant names with \textit{-amon} are clear loanwords; the suffIx \textit{-amon} is known from Pre-Greek), ultimately of uncertain origin, 4th c. BC; cf. cognates classical Latin \textit{cardamum}
    \parencites[s.v. cardamom]{oed}[s.v. cardamome]{tlfi}[s.v. cardamomum]{lewis_1879_latin}[s.v. καρδάμωμον]{liddell_1940_greekenglish}[s.v. κάρδαμον]{liddell_1940_greekenglish}[644]{beekes_2010_etymological}
\end{quote}

Kárdamon was identified with the word 𐀏𐀅𐀖𐀊 ka-da-mi-ja 41 , (kardamia as a feminine form of kardamon) appearing on Mycenaean tablets listing spices in Linear B, excavated in the “House of the Sphinxes” in 1950s, and dated to the 1200s bc (Bennett et al., 1958, p. 107).



this is how linguists usually reconstruct word histories: tracing word stages step by step. Basically cardamom came via Old French and Latin from a Greek word. I said kinda, because the exact origins are uncertain. And may or may not appear on Mycenaean stone tablets written in Linear B over 3000 years ago, it is outside of my specialty to judge these claims. It is quite difficult to be sure about the source of a word at this time-depth, almost 3000 years ago.

What about the etymology of doukou? Well, we had some suspicions, and during our investigations we came across various pieces of evidence that led us believe that doukou is a loanword. I will now introduce our observations and reasons why we think so in 4 points.

\subsection{First mentions, first confusions}

The first recorded mention of \tc{豆蔻} \textit{dòukòu} is from a 9th-century book called \tc{酉陽雜俎} \textit{Youyang Zazu} [\textit{Miscellaneous Morsels from Youyang}], which is a Tang era miscellany of tall tales and legends, strange phenomena, fantastic creatures, and exotic products -- but also an excellent source of historical data. It was collated by Duan Chengsi (d. 863), and in \hl{``chapter'' (juan (scroll, or book))} 18 he discusses 24 foreign plants, which have been imported to China or have been brought as tribute from faraway places, such as Magadha (in India), Malaysia, Persia, Silla (Korea), and Syria, often reporting the local names for the non-native plants and products, and usually compares them to something more familiar to his readership. We can find descriptions of acacia, Balm of Giliad, galbanum, jackfruit, jasmine, and Narcissus, among others \parencite{reed_2003_tang}. Section 55 tells us about cardamom, the text is accessible via the Chinese Text Project\footnote{\url{https://ctext.org/wiki.pl?if=en&chapter=801324}} \parencite{sturgeon_2021_chinese}, the translation is from us.



\begin{quote}
    \tc{\textbf{白豆蔻},出\textcolor{OliveGreen}{伽古羅}國,呼為\textcolor{OliveGreen}{多骨}。\\
    形如芭焦,葉似杜若,長八九尺,冬夏不凋。\\
    花淺黃色,子作朵如蒲萄。其子初出微青,熟則變白,七月採。}
    
    \textbf{White cardamom}, comes from the country of \textcolor{OliveGreen}{Kakula}, called \textcolor{OliveGreen}{/tɑ-kuət̚/}. [\dots]

    \begin{flushright}
        \addvspace{-2ex}
        (YYZZ §18:55)
    \end{flushright} 
\end{quote}

After stating the place of origin and its name, the author then proceeds to describe the plant's morphology: its height, its leaves, its yellow flowers, compares the shape of the fruits to grapes (also a \hl{foreign} plant in China), and puts the time of harvest to the seventh month of the lunar calendar.

\hl{...Middle Chinese reconstructions}

The fist obvious question here is: Why is it ``white'' cardamom? Or to put it more precisely, why does cardamom already have a modifier when it is the first attested instance we have of this word? According to \textcite[22]{donkin_2003_east}, the Chinese first confused nutmeg and cardamom, ``doubtless on account of a resemblance between their fruits''. We also know, that in the earliest sources, both spices were referred to as \textit{dòukòu} \parencites{hsu_1967_notes}{donkin_2003_east}, and that both were sourced from mainland Southeast Asia, and carried up to the Tang courts on ships from Kakola \parencite[184-185]{schafer_1985_golden}. This place also appears in Ibn Battuta \parencite{dunn_1986_adventures}


To avoid nomenclatural confusion, nutmeg became 肉豆蔻 ròudòukòu, cardamom became 白豆蔻 báidòukòu.
This mix-up exists in other languages as well!

According to Donkin, the Chinese first confused nutmeg with cardamom, on account of their similar fruits, and at some point both imported spices were called doukou.
Furthermore, both were sourced from mainland Southeast Asia, likely traded on the same trade routes, and the same ships, ((He says that, nutmeg was known in Chinese as kakola (ca. 725), and later as doukou (ca. 863), roudoukou is the name in later sources, including an illustrated herbal of 1062 (1249).))
So, to avoid confusion nutmeg became roudoukou – flesh cardamom – while the round cardamoms became baidoukou – white cardamom.
As many scholars noticed before, the confusion is not limited to Chinese.

((Nutmeg: “chia-kou-le” (ca. 725), as “to-ku” (ca. 863). “jou-tou-k’ou” (ca. 1062)
White cardamom: “tou-k’ou”, “pai-tou-k’ou” (Hsü, 1967; Donkin, 2003)))



\subsection{Character characteristics}

Our word under scrutiny is made up of two characters, \tc{豆} \textit{dòu} and \tc{蔻} \textit{kòu}. \textit{Dòu} is relatively straightforward, in dictionaries you can find definitions, such as `bean'; `pod-bearing plant or its seeds'; `bean-shaped object'; etc. ?? \textit{Kòu} on the other hand is much more specific, and dictionary entries usually direct the reader to other entries where the character is used, e.g., `used in 豆蔻'; `see豆蔻 nutmeg, cardamom'; etc. ?? The case in point here is that \textit{kòu} does not mean anything else, and it does not appear in other word, which is rare for a Chinese character. 

\tc{蔻} \textit{kòu} is made up of the radical for grass, \tc{艹} \textit{cǎo} `grass, herb' ??, and a phonetic component \tc{寇} \textit{kòu} meaning `bandit' which is a typical phono-semantic compound in Chinese. This sinogram however does not seem to exist before its emergence in the word for the spices, there is no record of \tc{蔻} \textit{kòu} before the 9th century. Was this character created for this purpose? It seems so. And if yes, then where does the \hl{/kou/ sound} come from? We think it is likely a loanword.

Furthermore, doukou often appears in a form ?? where the first character too has the grass radical on top (i.e., \tc{荳蔻}). Featuring the grass radical on both characters seems to be a typical device in the naming foreign edible plants, often loanwords themselves. Cf. the Chinese words \tc{草莓} \textit{cǎoméi} `strawberry', \tc{菠菜} \textit{bōcài} `spinach', and an archaic\footnote{The modern word for cumin, \tc{孜然} \textit{zī​rán} comes via Uyghur, \hl{from Sanskrit?}} word for `cumin', \tc{蒔蘿} \textit{shíluó} `cumin'.

\subsection{Lexicographical clues}

Youwei Shi (2021) Loanwords in the Chinese Language. Routledge:
	“Doukou 豆蔻, meaning cardamom, introduced to China in the 	Tang dynasty, probably originated from Arabic takur, related to 	the name of the ancient port Takola.” (p. 44)
No Arabic word like takur
Points to a place: Takola.

---

So, I tried to check what the dictionary says. Chinese is one the few major languages with no authoritative etymological dictionary, but there are some works on loanwords in Chinese. The most recent publication on this topic has an entry on doukou, and it says: 
…
This sounds like an crucial claim to push our investigation forward, 
but sadly, Arabic does not have a word that sounds or looks like takur
it does on the other hand point to an ancient port: Takola, which sounds awfully familiar to Kakola.





\subsection{The sea of kakkola/takkola type words}

Arabic قاقلة qāqulla ‘cardamom’ < Aramaic < Akkadian < Sanskrit?
Sanskrit कक्कोल kakkola ‘species of plant (bearing a berry, the inner part of which is waxy and aromatic’/ तक्कोल takkola *‘Pimenta acris’ (Monier-Williams, 1899, pp. 241, 431)
Pali takkola ‘perfume made from an aromatic berry’ (Pali Text Society, 1921–1925, p. 292),
Tibetan ཀ་ཀོ་ལ kakola ‘black cardamom’ (Amomum tsao-ko) (Goldstein et al., 2001, p. 1), 嘎哥拉 gágēlā in Eastern Tibet (Hu, 2005)
Javanese ꦏꦥꦸꦭꦒ kapulaga?, Malay pelaga? …etc.
Kakola (Chinese: Kakula, Arabic: Qāqulla) was a trading emporion on the western coast of Malay peninsula known in Ancient Greece as Takola of the golden Chersonese in Ptolemy’s Geography, AD 2nd c., and Talaitakkōlam in Tamil based on inscriptions from Chōla expeditions, AD 10th c. (Wheatley, 1961, p. 270).

---

There is an Arabic word for cardamom, qāqulla, which is not a native word. It is usually traced back to Sanskrit via Aramaic and Akkadian.
Looking up the Sanskrit word will send us down a rabbit hole about kakkola/takkola, a word that is the proposed etymon for others, such as Pali takkola, Tibetan kakola, and many more, including Javanese and Malay.
Kakola has been identified as Takola, an old trading emporion on the western coast of the Malay peninsula, already known to Ancient Greek geographers like Ptolemy, and also recorded in Tamil inscriptions of Chola expeditions.
The 4th reason therefore is about a regional review of seemingly related words, the technical word for these are Wanderwort of Kulturwort using German. 


(Just as the use of cinnamon may have led to the discovery of cloves, so one or other of the many cardamoms grown in South China and mainland South East Asia probably predisposed the Chinese to the imported nutmeg Takkola (Chinese Ko-ku-lo, Arabic) was a place or region on the west coast of Malaya,170 which the Chinese from the eighth century thought produced both nutmegs and the round cardamom (Amomum kepulaga), the latter possibly introduced from Java.171)



Other people have already noticed this before. Here is a collection of all the takkola/kakkola type words around the Indian Ocean world; the words usually have the sense of some aromatic substance or spice. 
Historians place the ancient port of Takola here, marked with the red dot, and we know from Chinese sources that the Chinese have imported products from here, so
Now the question is: Could 豆蔻 dòukòu be an eponymous loan related to one of these names? Could this transmission happen?






\printbibliography

\end{document}
